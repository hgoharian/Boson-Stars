\documentclass[12pt]{article}

\usepackage[utf8]{inputenc}

\usepackage[english]{babel}

\usepackage{amsthm}

\usepackage{tensor}

\usepackage{graphicx}

\usepackage{amsmath}

\usepackage{amssymb}

\usepackage{esvect}

\usepackage{setspace}

\usepackage[round]{natbib}

\usepackage{capt-of}

\usepackage{algorithm}

\usepackage[font=small,labelfont=bf]{caption}

\usepackage{color,soul}

\DeclareSymbolFont{matha}{OML}{txmi}{m}{it}

\DeclareMathSymbol{v}{\mathord}{matha}{118}

\usepackage{authblk}

\usepackage{ dsfont }

\usepackage{xcolor}

\renewcommand\thefootnote{\textcolor{red}{\arabic{footnote}}}

\newcommand*{\undercom}[3][gray]{\color{#1}\underbracket[0.5pt][4pt]{\normalcolor#2}_{#3}\normalcolor}


\setlength{\topmargin}{-1in}

\setlength{\headheight}{1.5cm}

\setlength{\headsep}{0.5cm}

\setlength{\textheight}{8.9in}

\setlength{\oddsidemargin}{.1in}

\setlength{\evensidemargin}{.1in}

\setlength{\textwidth}{6.3in}


\usepackage{mathtools,bm,amsfonts,amssymb,lmodern}

\usepackage{mathdesign}

\DeclareMathOperator{\var}{var}

\renewcommand{\vec}[1]{\bm{#1}}


\numberwithin{equation}{section}


\newtheorem{theorem}{Theorem}

\numberwithin{theorem}{subsection}

\newtheorem{example}[theorem]{Example}

\newtheorem{definition}[theorem]{Definition}

\newtheorem{exercise}[theorem]{Exercise}

\newtheorem{proposition}[theorem]{Proposition}

\newtheorem{note}[theorem]{Note}

\newtheorem{lemma}[theorem]{Lemma}

\newtheorem{corollary}[theorem]{Corollary}

\newtheorem{question}[theorem]{Question}

\newtheorem{project}[theorem]{Project}

\newtheorem{problem}[theorem]{Problem}

\newtheorem{conjecture}[theorem]{Conjecture}

\newtheorem{remark}[theorem]{Remark}






\newcommand{\N}{\mbox{$I\!\!N$}}

\newcommand{\Z}{\mbox{$Z\!\!\!Z$}}

\newcommand{\Q}{\mbox{$I\:\!\!\!\!\!Q$}}

\newcommand{\R}{\mbox{$I\!\!R$}}

\newcommand{\defeq}{\vcentcolon=}


\title{Boson Stars}

\author{Helia Goharian }

\date{}



\begin{document}


\maketitle

\section{Contents}

\section{Introduction}

Boson stars come under the category of exotic stars that are yet still theoretical due to the difficulty of testing in detail how forms of matter may behave. Prior to the emerging technology of gravitational-wave astronomy, there was no sufficient way of detecting cosmic objects that do not radiate electromagnetically or through known particles. Such objects are occasionally identified based on indirect evidence gained from observable properties. 

\newline

A boson star is a hypothetical astronomical object that is formed out of particles called bosons. It is theorized that unlike normal stars, which emit radiation due to gravitational pressure and nuclear fusion, boson stars would be transparent and invisible. The extremely large gravity of a compact boson star would bend light around the object, creating an empty region resembling the shadow of a black hole's event horizon.

\newline

Computational simulations further suggest that rotating boson stars would be doughnut-shaped as centrifugal forces would give bosonic matter that form. As of 2018, it may become possible to detect them by the gravitational radiation emitted by a pair of co-orbiting boson stars.


\section{Preliminary Matter}

The mathematical language of general relativity is mostly based on differential geometry. In this chapter, we recall some basic definitions (without proof) that I will need and refer back to through out the rest of the paper. To clarify some notation, $M$ denotes the generic manifold of the 4-dimensional space and $\vec{g}$ is the metric on $M$.

\begin{itemize}

    \item General Relativity models the universe as a 4-dimenstional manifold $M$, called the \textbf{\textit{spacetime manifold}} where points on the manifold are referred to as \textit{events}.

    \item The spacetime manifold is equipped with a Lorentzian metric $\tensor{g}{_\mu_\nu}$ $(-,+,+,+)$.

    \item Assigned to the 4-dimenstional manifold $M$ is the spacetime metric $\tensor{g}{_\mu_\nu}$ which satisfies \textbf{Einstein's Field Equations}:

    \begin{center}

\boxed{$\tensor{R}{_\mu_\nu}-\frac{1}{2}R\tensor{g}{_\mu_\nu} +\Lambda \tensor{g}{_\mu_\nu}=8\pi G \tensor{T}{_\mu_\nu}$}

\end{center} This significant equation, published by Einstein in 1915, describes the association between the geometry of spacetime and the distrubution of matter within it.


\item\begin{definition*}\textbf{(p-linear alternating forms):} Let $\Lambda^{p}v$ be the vector space of p-linear alternating forms with real values: $$\Lambda^{p}v := \{\varphi:v \times v \times .... \times v &\to \mathbb{R},multilinear, alternating\}$$

where we have p factors of v. 

\end{definition*} \newline The set of the p-linear alternating forms at the point $p$, sets up an $n$-dimensional vector space, which is called the \textbf{dual tangent space} and is denoted by $T^{*}_{p}(M)$. Our natural basis, ($\vec{\partial}_{\mu}$),  is the tangent space $T_{p}(M)$ at point $p$ associated with some coordinate ($x^{\mu}$). \smallskip

\newline \textbf{Note:} We have $\langle \vec{d}x^{\mu},\vec{\partial}_{\nu}\rangle=\tensor{\delta}{^\mu_\nu}$, where $\delta^{\mu}_{\nu}$ is the \textit{Kronecker symbol} i.e. $\tensor{\delta}{^\mu}{_\nu}=1$ when $\mu$ = $\nu$ and $0$ otherwise.

\item \textbf{Covariant derivative:} The covariant derivative of a tensor field $\vec{T}$ of type $(l,k)$ is a tensor field $\vec{\nabla T}$ of type $(l,k+1)$ where in a natural fram associated with some coordinate system $(x^{\mu})$ we have:

\begin{align*}

    \nabla_{\lambda}\tensor{T}{^{\mu_{1}}^\dots^{\mu_{l}}_{\nu_{1}}_\dots_{\nu_{k}}} = (\partial_{\lambda}x)\, \tensor{T}{^{\mu_1}^\dots^{\mu_l}_{\nu_1}_\dots_{\nu_k}}+ \sum_{i=1}^{l}\tensor{\Gamma}{^{\mu_i}_\alpha_\lambda} \tensor{T}{^{\mu_{1}}^\dots^{\alpha}^\dots^{\mu_l}_{\nu_1}_\dots_{\nu_k}} - \sum_{i=1}^{k}\tensor{\Gamma}{^\alpha_{\nu_{i}}_\lambda}\tensor{T}{^{\mu_{1}}^\dots^{\mu_{l}}_{\nu_{1}}_\dots_\alpha_\dots_{\nu_{k}}} 

   \end{align*}

    Given $\vec{v} \in T_{p}(M)$, $\nabla{\vec{v}}$ is called the covariant derivate of $\vec{v}$ with respect to the \textit{\textbf{affine connection $\nabla$}}. Let's next list some properties of this affine connection on $M$:

    \begin{enumerate}

        \item $\nabla$ is bilinear.

        \item For any scalar field $f$ and pair of vectors $(\vec{u},\vec{v}) \in T_{p}(M)$: $$\nabla_{f \vec{u}}\vec{v} = f \nabla_{\vec{u}}\vec{v}$$

        \item  For any scalar field $f$ and pair of vectors $(\vec{u},\vec{v}) \in T_{p}(M)$ the following Leibniz rule holds: $$\nabla_{\vec{u}}(f\vec{v}) = \langle \nabla f , \vec{u} \rangle \vec{v} + f \nabla_{\vec{u}}\vec{v}$$ 

    \end{enumerate}

\item \textbf{Christoffel Symbols:}

In this paper we will only consider the unique affine connection ($\vec{\nabla}$), called the Levi-Civita connection associated with the metric $\vec{g}$ on our manifold $M$. Given a coordinate system $x^{\mu}$ on $M$ we define the connection coefficients of the Levi-Civita connection with respect to the basis $\vec{\partial}_{\mu}$ as the \textbf{\textit{Christoffel symbols}}:

$$\tensor{\Gamma}{^\alpha_\mu_\nu} = \frac{1}{2} \tensor{g}{_\alpha_\lambda} \left( \frac{\partial \tensor{g}{_\lambda_\nu}}{\partial \tensor{x}{^\mu}} + \frac{\partial \tensor{g}{_\mu_\lambda}}{\partial x^{\nu}} - \frac{\partial \tensor{g}{_\mu_\nu}}{\partial x^{\beta}}  \right)$$

It is good to keep in mind that the Christoffel symbols are symmetric with respect to the lower two indices. 

The connection indicates the rate at which the basis vectors are changing at each point in space.

In a manifold $(M,\vec{g})$ the unique affine connection $\nabla$ is \begin{enumerate}

    \item \textbf{\textit{torsion-free}} meaning  for any scalar field $f$ we have that: $$\nabla_{\mu}\nabla_{\nu}f = \nabla_{\nu}\nabla_{\mu}f$$

    \item The covariant derivative of the metric tensor vanishes:

    $$\vec{\nabla g} = 0$$

\end{enumerate}

\item \textbf{Riemann curvature tensor:} In space of higher dimension we require more than one quantity at each point to describe curvature. It turns out that the right definition involves the components of a 4-index tensor called the \textbf{\textit{Riemann curvature tensor}}. In a coordinate basis, the components of the Riemannian tensor are given in terms of connection coefficients:

$$\tensor{R}{^\alpha_\beta_\mu_\nu} = \frac{\partial \, \tensor{\Gamma}{^\alpha_\beta_\nu}}{\partial x^{\mu}} - \frac{\partial \, \tensor{\Gamma}{^\alpha_\beta_\mu}}{\partial x^{\nu}} + \, \tensor{\Gamma}{^\alpha_\sigma_\mu} \tensor{\Gamma}{^\sigma_\beta_\nu} -  \tensor{\Gamma}{^\alpha_\sigma_\nu} \tensor{\Gamma}{^\sigma_\beta_\mu}$$

It can be checked that \[\tensor{R}{_\mu_\nu_\alpha_\beta} = -\tensor{R}{_\nu_\mu_\alpha_\beta}, \; \; \; \tensor{R}{_\mu_\nu_\alpha_\beta} = - \tensor{R}{_\mu_\nu_\beta_\alpha}, \; \; \; \tensor{R}{_\mu_\nu_\alpha_\beta} = \tensor{R}{_\alpha_\beta_\mu_\nu} \]

The symmetries amount to 236 constraints and so $\tensor{R}{^\alpha_\beta_\mu_\nu}$ has only 20 non-zero components.

\item\textbf{Ricci identity:} 

From the properties of the Riemann curvature follows the non-commutative \textbf{\textit{Ricci identity}}, which is satisfied by the covariant derivatives of second order with respect to the metrci $\vec{g}$:

\begin{equation}

    \forall v \in T(M), \; \; 

    (\nabla_{\alpha}\nabla_{\beta}-\nabla_{\alpha}\nabla_{\beta})v^{\lambda} = \tensor{R}{^\lambda_\mu_\alpha_\beta}\tensor{v}{^\mu}

\end{equation}

\item \textbf{Ricci tensor:} The Ricci tensor is obtained by contracting the first and third indices of the Riemann tensor:

$$\tensor{R}{_\mu_\nu} \equiv \tensor{g}{^\alpha^\beta}\tensor{R}{_\alpha_\mu_\beta_\nu} = \tensor{R}{^\alpha_\mu_\alpha_\nu}$$

Because of the symmetries of the Riemann tensor one has that the Ricci tensor is symmetric:

$$\tensor{R}{_\mu_\nu} = \tensor{R}{_\nu_\mu}$$

The \textbf{\textit{Ricci scalar}} is defined as the contraction of the Ricci or Riemann tensor indices:

$$R \equiv \tensor{g}{^\mu^\nu}\tensor{R}{_\mu_\nu} = \tensor{g}{^\mu^\nu}\tensor{g}{^\alpha^\beta}\tensor{R}{_\mu_\nu_\alpha_\beta}$$

\item\textbf{Lie derivative:}

Imagine a fast flowing river with two neighbouring fluid elements inside it. Furthermore, imagine a vector connecting the two neighbouring elements which measures their separation. What would happen to this vector as our pair of neighbouring elements follow the streamline (i.e. flow) of the river? \newline

Let's now turn our heads back to our manifold $M$ on which we define a vector field $u^{i}(x)$. Consider the integral curves of this vector field given by $x^{i}(\mu,\lambda)$ where the curves are tangent to our vector field at each point:

\begin{equation}

    \frac{\partial x^{i}}{\partial \lambda} = u^{i}(x)

\end{equation}

\end{itemize}

\begin{figure}[htp]

\centering

\includegraphics[width=10cm]

{"Lie derivative".jpg}

\begin{center}

  \textbf{Figure 0}

\end{center}

\label{fig:fig}

\end{figure}  As shown schematically in \textbf{figure 0}, first note that each point on the manifold only has \underline{one} curve passing through it. Moreover on curve $\mu$ we have points $P(x)$ and $Q(x+dx)$ where they are infinitesimally separated and at each point we have vector fields $v^{i}(x)$ and $v^{i}(x+dx)$ accordingly. What we want to find out is how a vector on $P(x)$ changes as it moves along the curve $\mu$ towards point $Q$. Let $Q(x+dx)=Q(x')$ and $x^{i}$ be a coordinate system on our manifold $M$. We'll represent what's happened to the vector $v^{i}(x)$ at point in terms of its value starting from point P. \newline To begin, we'll use the following infinitesimal coordinate transformation: $$x'^{i} = x^{i} + dx^{i}$$Making use of Eq.(3.2) the above becomes:

\begin{equation}

    x'^{i} = x^{i} + u^{i}d\lambda 

\end{equation}

Where $u^{i}$ is our tangent vector at point $P(x)$ in Eq.(3.3). The change of coordinates of a vector field $$v'\,^{i}(x') = \frac{\partial x'\,^{i}}{\partial x^{j}} v^{j}(x)$$ leads us to the following:

\begin{align}

    v'\,^{i}(x') &= \frac{\partial x'\,^{i}}{\partial x^{j}} v\,^{j}(x) \notag \\

    &= \left(  \frac{\partial x^{i}}{\partial x^{j}} + \frac{\partial u^{i}}{\partial x^{j}}d\lambda \right) v\,^{j}(x) \notag \\

    &= (\tensor{\delta}{^i_j} + \partial_{j} u^{i} d\lambda)\, v\,^{j}(x) \notag \\

    &= v\,^{i}(x) + \partial_{j} u^{i} v\,^{j}(x) d\lambda

    \end{align}

As $x^'{i}$ and $x^{i}$ represent points P and Q on curve $\mu$ we get the \textit{Lie dragged} vector from point P to Q:

\begin{equation}

    \boxed{v' \,^{i}(Q) = v\,^{i}(P) + \partial_{j} u^{i} v\,^{j}(P)d\lambda}

\end{equation}

We also have for vector $v \,^{i}$:

\begin{equation}

    \boxed{v \, ^{i}(Q) = v\,^{i}(x+dx) &\approx v\,^{i}(x) + dx^{j} \, \frac{\partial v\,^{i}(x)}{\partial x^{j}}\notag  &\approx v\,^{i}(P) + u^{j} \partial_{j} v^{i}(P)d\lambda}

\end{equation}

Where we've used the fact that $f(x+dx) \approx f(x) + f'(x)dx$.

\newline We now evaluate the change of the vector $v\,^{i}$ along the tangent vector u at point Q by using its value at point P, giving us the \textit{Lie derivative} of the vector $v\,^{i}$ along the curve $\mu$:

\begin{align}

    \mathcal{L}_{u} v\,^{i}(P) &= \lim_{d\lambda \to 0} \frac{v \, ^{i}(Q) - v'\,^{i}(Q)}{d\lambda} \notag \\[6pt]

    &= u^{j}\partial_{j} v^{i}(P) - v^{j}(P)\partial_{i}u^i

\end{align}

We can also re-write the above as:

\begin{equation}

    \boxed{\mathcal{L}_{u} v\,^{i} = u^{j} \nabla_{j} v \, ^{i} - v \,^{j} \nabla_{i} u^{i}}

\end{equation}

Due to the reason that our calculation is being performed on a \underline{single point} and so the Christoffel symbols cancel out.

\newline

In adapted coordinates,i.e., $x^{\mu} = (t, x^{i})$ the Lie derivative is simply obtained by taking the partial derivative of the vector with respect to $t$:

\begin{equation}

    \mathcal{L}_{\vec{u}}v^{\mu} = \frac{\partial v^{\mu}}{\partial t}

\end{equation}

It is now possible to conclude that at point P the manifold is being shipped forward to point Q. However, we are interested in how to find the Lie derivative of tensors. Not to worry, finding the Lie derivative of a mixed tensor (of the form $\tensor{T}{^\mu_\nu}$) follows the same steps as we took for the vector $v^{i}$. Before we jump into that we shall first outline two properties of the Lie derivative:

\begin{enumerate}

    \item If $\mathcal{L}_{u} v\,^{i} = 0$, then we say that the vector has been \textit{Lie transported}.

    \item The Lie derivative of a scalar function $f$ is $$\mathcal{L}_{u} f = \frac{df}{d\lambda} = u^{i}\nabla_{i}f$$

\end{enumerate}

Now how about the Lie derivative of a tensor $\tensor{T}{^\mu_\nu}$?

\newline Following the same procedure and coordinate transformations as what we did for the Lie derivative of a vector, we get the final result:

$$\mathcal{L}_{u} \tensor{T}{^i_j} = u^{k}\nabla_{k}\tensor{T}{^i_j} + \tensor{T}{^i_l} \nabla_{j} u^{l} - \tensor{T}{^k_j} \nabla_{k} u^{i} $$

Where we can generalise for the Lie derivative of any rank tensor:

\begin{align}

    \mathcal{L}_{u} \tensor{T}{\,^a^b^c^\dots_d_e_f_\dots} &= u^{g} \partial_{g} \tensor{T}{^a^b^c^\dots_d_e_f_\dots}  - \partial_{g} u^{a}  \tensor{T}{^g^b^c^\dots_d_e_f_\dots} - \notag \\ &- \partial_{g} u^{b} \tensor{T}{^a^g^c^\dots_d_e_f_\dots} - \partial_{g} u^{c} \tensor{T}{^a^b^g^\dots_d_e_f_\dots} - \dots \notag \\

    &+ \partial_{d} u^{g} \tensor{T}{^a^b^c^\dots _g_e_f_\dots} + \partial_{e} u^{g} \tensor{T}{^a^b^c^\dots _d_g_f_\dots} + \dots

    \end{align}




\section{What is a boson star}

Boson stars (BS) are assembled with a complex scalar field coupled to gravity.

A complex scalar field $\Phi(t,r)$ can be decomposed into two real scalar fields $\Phi_{R}$ and $\Phi_{I}$ which satisfy an $SO(2)$ symmetry and map every spacetime event to a complex plane. We have that:

\begin{center}

    $\Phi(t,r) \equiv \Phi_{R}(t,r)+i\Phi_{I}(t,r)$

\end{center}

Such a field possesses energy because of its spatial gradients and time derivatives. This energy holds the star together. The real question here is what carries the star against the force of gravity. The relative scalar field also obeys a \textit{Klein-Gordon} wave equation which tends to disperse fields and hence should lead to the dispersion of the star. Therefore considering the system of gravity coupled with scalar fields gives us a range of spherically symmetric solutions. When the scalar fields are non self-interacting one finds the BS solutions which are prevented from collapsing following the Heinsberg uncertainty principle. In a real scalar field, there are no regular and non-singular, static solutions therefore through out this paper we will only be focusing on complex scalar fields. 

\newline

We assume the boson star to be a macroscopic quantum state and after applying the uncertainty principle one can find an excellent estimate for the maximum mass of a BS. We first begin by stating the Heinsberg uncertainty principle of quantum mechanics \begin{center}

 $\bigtriangleup p \bigtriangleup x \geq \hbar$ 

\end{center}and take the BS to be enclosed within radius $\bigtriangleup x = R$ with a maximum momentum of $\bigtriangleup p = mc$ where $m$ is the mass of the particle.

This results in \begin{equation}

    mc R \geq \hbar

    \end{equation}

 To look find the maximum possible mass $M_{max}$ for the BS the uncertainty bound is eliminated and the radius of the star is taken towards its Schwarzschild radius $R_{s} \equiv 2GM/c^2$. Substituting in Eq. (1) yields \begin{equation}

     \frac{2 Gm  M_{max}}{c} = \hbar

 \end{equation} and consequently we find an expression for the maximum mass \begin{equation}

     M_{max} = \frac{1}{2} \frac{\hbar c}{Gm}

 \end{equation}

 We also have that the Planck mass $M_{Planck} \equiv \sqrt{\hbar c/G}$, we derive the estimated maximum mass of a BS to be \begin{equation}

     M_{max}=\frac{0.5M^2_{Planck}}{m}

 \end{equation}

 This estimate tells us that the maximum mass of the BS is inversely in relation to the mass of the constituent scalar field. 

 Note that depending on the strength of the coupling parameter $m$ as well as other parameters of the self-interacting potential, the size and mass of the boson stars can vary from atomic to astrophysical scale.

 



\section{Conventions}

Throughout this paper, Greek letters ($\mu,\nu,\alpha,\beta,...$) indicate spacetime indices ranging from 0 to 3 whilst Roman letters ($i,j,k,l,...$) ranging from 1 to 3 denote spatial indices. We will also be using units such that $\hbar = c = 1$ and the signature convention $(-,+,+,+)$ for the metric.



\section{Hypersurfaces in Spacetime}

The elegant and simple appearance of Einstein's equations hides a great deal of complexity. In fact when expanded in tensor notation it can contain up to a thousand terms and will have significant non-linearities which will make it difficult to directly solve for. 

\newline

The 3+1 formalism originated from studies by Georges Darmois in the 1920s, Andr\'e Lichnerowicz in the mid 1930's and Yvonne Choquet-Bruhat in the 1950's is an approach to general relativity which remodels Einstein's equations in computable,time-step iteration processes that would evolve from initial conditions to a final state. The main idea behind this approach is to decompose space and time by stacking 3D slices where each slice is characterised by a fixed time coordinate. The evolution of each slice is given by EFE. We first start by defining a hypersurface.

\subsection{Definition of Hypersurfaces}

\begin{definition}

A hypersurface of $M$ is the image of a 3D manifold labelled as $\hat{\Sigma}$ by an embedding $\phi:\hat{\Sigma} \rightarrow{M}$ thus

\begin{center}

    $\Sigma = \phi(\hat{\Sigma})$

\end{center}

\end{definition}

An embedding is a homeomorphic map, in this case $\phi: \hat{\Sigma} \rightarrow \Sigma$. Meaning $\phi$ is one-to-one and both $\phi$ and $\phi^{-1}$ are continuous. The one-to-one property ensures that $\Sigma$ does not intersect itself. Now to introduce a local coordinate system which is defined on $\Sigma$ we first define the following:

\newline

A hypersurface can be defined locally as a set of points such that a scalar field on $M$ is constant: \begin{equation}

    \forall p \in M, p \in \Sigma \iff t(p)=0 

\end{equation}

Let $x^{\mu}=(t,x,y,z)$ be a local coordinate system of $M$ where $(x,y,z)$ are Cartesian coordinates spanning $\mathbb{R}^{3}$. According to Eq. (6.1), $\Sigma$ is defined by the coordinate condition $t=0$ and the mapping $\phi$ can be obtained by taking $x^i=(x,y,z)$ to be the coordinates on the 3D-manifold $\hat{\Sigma}$ :

\begin{align*}

\phi \colon \hat{\Sigma} &\to M \\ 

(x,y,x) &\mapsto (0,x,y,z)

\end{align*}

The embedding $\phi$ "transmits" curves in $\hat{\Sigma}$ to curves in $M$. Consequently $\phi$ also "transmits" the vectors in $\hat{\Sigma}$ to vectors in $M$ resulting in a mapping between the tangent spaces of $\hat{\Sigma}$ and $M$.

\begin{figure}[htp]

\centering

\includegraphics[width=11cm]

{"push-forward".png}

\begin{center}

  \textbf{Figure 1:}Adapted from "3+1 Formalism and Bases of Numerical Relativity" Lecture Notes by Eric Gourgoulhon

\end{center}

\label{fig:fig}

\end{figure}

\newline In \textbf{figure (1)} we have the push-forward $\phi_{*}$ of a vector $v$ tangent to some curve $c$ is a vector tangent to $\phi(c)$ in $M$.

In result we define the following:

\theoremstyle{definition}

\begin{definition}

The push-forward mapping is defined as follows:

\begin{align*}

\phi_{*} \colon T_{p}(\hat{\Sigma}) &\to T_{p}(M) \\

v = (v^{x},v^{y},v^{z}) &\mapsto \phi_{*}v = (0,v^{x},v^{y},v^{z})

\end{align*}

where $T_p(\hat{\Sigma}) \, \textrm{and} \, \, T_p(M)$ are the corresponding tangent spaces in $\hat{\Sigma} \, \textrm{and} \, \, M$ with respect to the basis $\vec{\frac{\partial}{\partial x^{\mu}}}$ associated with the coordinate $x^{\mu}$ and so for the tangent vector at point $p$ we have $v^{\mu}(p)=\vec{\partial_{\mu}}v(p)$.

\end{definition}

Conversely, we can also define the \textit{pull-back} mapping between the linear forms living in the cotangent spaces of $\hat{\Sigma}$ and $M$ at point $p$:

\begin{align*}

    \phi^{*} \colon T_{p}^{*}(M) &\to T_{p}^{*}(\hat{\Sigma}) \\

    \underline{\vec{\omega}} = (\omega_{t},\omega_{x},\omega_{y},\omega_{z}) &\mapsto \phi^{*}\underline{\vec{\omega}} = (\omega_{x}, \omega_{y}, \omega_{z})

\end{align*}

where $\omega_{\mu}$ denotes the components of the 1-form $\underline{\vec{\omega}}$ with respect to the basis $dx^{\mu}$.

\newline

Naturally the the pull-back operation can be extended to multi-linear forms on $T_{p}(M)$ as follows:

\begin{definition}

If $\vec{S}$ is a n-linear form on $T_{p}(M)$ then we have that $\phi^{*}\vec{S}$ is the n-linear form on $T_{p}(\Sigma)$ defined by

\begin{align*}

    \forall(\vec{v_{1}},\vec{v_{2}},...,\vec{v_{n}}) \in T_{p}(\Sigma)^{n}, \, \, \phi^{*}\vec{S}(\vec{v_{1}},\vec{v_{2}},...,\vec{v_{n}}) = \vec{S}(\phi_{*}\vec{v_{1}},...,\phi_{*}\vec{v_{n}})

\end{align*}

\end{definition}

\subsection{The Normal vector}

Going back to our constant scalar field $t$ on $M$ [cf. Eq.(6.1)] $\Sigma$ is defined in such a way that it is in a level surface of $t$ resulting in every vector $\vec{v}$ to be tangent to $\Sigma$, i.e. $\langle \nabla t,\vec{v}\rangle=0$. The vector

$\overrightarrow{\nabla} t$, is the metric dual to our gradient 1-form $\nabla t$,

$$\nabla^{\alpha} t = \tensor{g}{^\mu^\nu} \bigtriangledown_{\mu} t$$

where $\overrightarrow{\nabla} t$ is normal to $\Sigma$ and hence defines the unique direction normal to it.  $\overrightarrow{\nabla} t$ also satisfies to the following properties

\medskip

\begin{itemize}

    \item  $\overrightarrow{\nabla} t$ is timelike iff $\Sigma$ is spacelike;

    \item  $\overrightarrow{\nabla} t$ is spacelike iff $\Sigma$ is timelike;

    \item  $\overrightarrow{\nabla} t$ is null iff $\Sigma$ is null.

\end{itemize}

We can re-normalise $\overrightarrow{\nabla} t$ to make it a unit vector: 

\begin{equation}

    n_{\mu}=\pm(\tensor{g}{^\mu^\nu}\nabla_{\mu}\nabla_{\nu}\, t)^{-\frac{1}{2}}\nabla_{\mu}\, t

\end{equation}

Where in index-free notation Eq.(6.2) can be written as:

$\vec{n} \defeq \pm (\overrightarrow{\nabla} t \cdot \overrightarrow{\nabla} t)^{-1/2} \, \overrightarrow{\nabla} t$, where the sign $+$ corresponds to a timelike hypersurface and the sign $-$ is for a spacelike one. \smallskip

It can be concluded that the vector $\vec{n}$ is a unit vector by constructing the following: 

\begin{align}

    \vec{n} \cdot \vec{n} &= -1, & \text{if} \; \Sigma  \; \text{is timelike} \notag\\

    \vec{n} \cdot \vec{n} &= 1, &  \text{if} \; \Sigma \; \text{is spacelike}  \notag

\end{align}

Since in our case, $\vec{n}$ is a timelike unit vector, it can be regarded as the 4-velocity of some observer which we shall refer to as the \textbf{\textit{Eulerian observer}}. Due to the orthogonality of $\vec{n}$ to our spacelike hypersurafes, it can be concluded that the worldlines of the Eulerian observers are also normal to the hypersurface. This means that according to the conventionality of Einstein's simultaneity, the hypersurface $\Sigma_{t}$ is \textit{locally} the set of events that are simultaneous from the Eulerian observer's point of view. 

\newline Coincidentally the 4-acceleration of the Eulerian observer can be deduced by taking the covariant derivative of $\vec{n}$ itself: $$\vec{a} \defeq \nabla_{\vec{n}} \vec{n}$$ $\vec{a}$ is orthogonal to $\vec{n}$ and hence tangent to $\Sigma$ since \begin{align}\vec{n} \cdot \vec{a} = \vec{n} \cdot \nabla_{\vec{n}} \vec{n}= \frac{1}{2} \nabla_{\vec{n}}(\vec{n \cdot n}) = \frac{1}{2} \nabla_{\vec{n}}(-1) = 0\end{align}

\smallskip

We will re-visit Eulerian observers in Chapter 7.

\subsection{Intrinsic Curvature}

Just like our metric $\vec{g}$ on $M$ the induced metric on our $\Sigma$, $\overrightarrow{\vec{\gamma}}$, on either the spacelike or timelike hypersurface is also not degenerate. This means that there is a unique connection (or covariant derivative) $\vec{D}$ on the manifold $\Sigma$ that is \textit{torsion free} and hence satisfies:

$$\boxed{\vec{D}\vec{\overrightarrow{\vec{\gamma}}}=0}$$

$\vec{D}$ is the Levi-Civita connection on our hypersurface associated with $\overrightarrow{\vec{\gamma}}$. As in the case of our 4D manifold, $M$, $\Sigma$ also has some curvature associated with the connection $\vec{D}$. The Riemann tensor associated with this connection is called the \textbf{\textit{intrisic curvature}} of $(\Sigma,\vec{\gamma})$ denoted by $\tensor{R}{^k_l_i_j}$. Our 3D Riemann curvature tensor measures the non-commutativity of two successive covariant derivatives. Let's remind ourselves of the \textit{Ricci identity} [cf. Eq.(3.2)] however this time on $(\Sigma,\vec{\gamma})$:

\begin{equation}

    \forall \vec{v} \in T(\Sigma), \; \; \; [D_{i},D_{j}]v^{k} = (D_{i}D_{j}-D{j}D{i})v^{k} = \tensor{R}{^k_l_i_j}v^{l}

\end{equation}

In order to derive the Ricci tensor and Ricci scalar on $\Sigma$, we follow the same principles as on $M$:

$$\tensor{R}{_i_j} = \tensor{R}{^k_i_k_j}$$ and $R = \tensor{\gamma}{^i^j}\tensor{R}{_i_j}$ where ($\vec{g} \rightarrow \vec{\gamma}$). The 3D Ricci scalar $R$ is also called the \textit{Gaussian curvature} of $(\Sigma,\vec{\gamma})$.

\subsection{Extrinsic Curvature}

In addition to the intrinsic curvature on our hypersurface, there is another type of "curvature" namely one which corresponds to the "bending" of $\Sigma$ in $M$. This bending is associated with the change of the direction of the unit normal vector $\vec{n}$ as one moves from one point to another on $\Sigma$. We define the \textbf{\textit{Weingarten map}} as the endomorphism of $T_{p}(\Sigma)$ which maps every vector $\vec{v} \in T_{p}(\Sigma)$ to the change of the normal along the tangent vector $\vec{v}$. In other words, the way of investigating the shape of a surface is to measure the change in the unit normal vector near to that point. This change is evaluated via the spacetime connection.

\begin{align*}

    \chi \colon T_{p}(\Sigma) &\to T_{p}(M) \\

    \vec{v} &\mapsto \nabla_{\vec{v}}\vec{n}

\end{align*}

From the above we conclude that $\chi(\vec{v}) \in T_{p}(M)$. \newline A fundamental property of the Weingarten map is to be \textit{self-adjoint} with respect to the induced metric $\vec{\gamma}$:

\begin{equation}

    \boxed{\forall (\vec{u},\vec{v}) \in T_{p}(\Sigma) \times \in T_{p}(\Sigma), \; \; \vec{u} \cdot \chi(\vec{v}) = \chi(\vec{u}) \cdot \vec{v}}

\end{equation}

We shall prove Eq.(6.5) using the property of Eq.(6.3) and  $n_{\mu} \defeq \alpha \nabla_{\mu}t$ where

\newline $\alpha \defeq \pm (\overrightarrow{\nabla}t \cdot \overrightarrow{\nabla}t)^{-1/2}$: 

\begin{align*}

    \vec{u} \cdot \chi(\vec{v}) - \chi(\vec{u}) \cdot \vec{v} &= \vec{u} \cdot \nabla_{\vec{v}}\vec{n} - \vec{v} \cdot \nabla_{\vec{u}}\vec{n} \\ 

    &= u^{\mu}v^{\nu}\nabla_{\nu}n_{\mu} - v^{\mu}u^{\nu}\nabla_{\nu}n_{\mu} \\

    &= u^{\mu}v^{\nu}(\nabla_{\nu}n_{\mu}-\nabla_{\mu}n_{\nu}) \\ &= u^{\mu}v^{\nu}[\nabla_{\nu}(\alpha \nabla_{\mu}t)-\nabla_{\mu}(\alpha \nabla_{\nu}t)] \\ &= u^{\mu}v^{\nu}[\nabla_{\nu}\alpha\nabla_{\mu}t+\alpha\nabla_{\nu}\nabla_{\mu}t-\nabla_{\mu}\alpha\nabla_{\nu}t-\alpha\nabla_{\mu}\nabla_{\nu}t] \\ &= u^{\nu}\nabla_{\nu} \alpha\, \underbrace{u^{\mu}\nabla_{\mu}t}_{=0} - u^{\mu}\nabla_{\mu}\alpha \,\underbrace{u^{\nu}\nabla_{\nu}t}_{=0} + \, \alpha \, u^{\mu}u^{\nu}\underbrace{(\nabla_{\nu}\nabla_{\mu}t-\nabla_{\mu}\nabla_{\nu}t)}_{= 0} \\ &= 0 \; \; \; \square

\end{align*}

Where we have used the Leibniz rule in the fifth line of our derivation and the fact that the connection $\nabla$ is torsion-free. Also since $\vec{u}, \vec{v} \in T_{p}(\Sigma)$ then $u^{\mu}\nabla_{\mu}t = 0$. 

\newline An interesting property of the Weingarten map is that due to its self-adjointness it has real eigenvalues and that these eigenvalues correpond to the eigendirections of $\Sigma$. These eigenvectors define the so-called \textit{principle directions} of $\Sigma$.

\smallskip

\newline \textbf{Remark:} The intrinsic curvature is a quantity \textit{independent} of the way the manifold $(\Sigma,\vec{\gamma})$ is embedded in $(M,\vec{g})$. In the contrary, the extrinsic curvature and its so-called principle directions depend on the embedding. \smallskip \newline

It's now time to define the \textbf{\textit{extrinsic curvature}} tensor of $\Sigma$ denoted as $\vec{K}$.

\begin{align}

    \vec{K} \colon T_{p}(\Sigma) \times &T_{p}(\Sigma) \to \mathds{R} \notag \\ 

    (\vec{u},\vec{v}) &\mapsto -u \cdot \chi(\vec{v})

\end{align}

The self-adjointness of $\chi$ implies that the bilinear form defined on $\Sigma$'s tangent space by (6.6) is symmetric.\newline Making the value of $\chi$ explicit we get the original definition of $\vec{K}$:

\begin{align}

    \forall (\vec{u},\vec{v}) \in \; T_{p}(\Sigma) \times T_{p}(\Sigma) \Rightarrow  

    \vec{K}(\vec{u},\vec{v}) = - \vec{u} \cdot \nabla_{\vec{v}}\vec{n}

\end{align}

We can explicitly derive Eq.(6.7) by using $\boxed{\tensor{K}{_\mu_\nu} = -\nabla_{\mu}n_{\nu}}$:

\begin{align*}

K(\vec{u},\vec{v}) &= \tensor{K}{_\mu_\nu}u^{\mu}v^{\nu} \\

&= -\nabla_{\mu}n_{\nu}u^{\mu}v^{\nu} \\

&= -(u^{\mu}\nabla_{\mu}n_{\nu})v^{\nu} \\

&= -(\nabla_{\vec{u}}\vec{n})\vec{v} \; \; \; \square

\end{align*}



\subsection{Orthogonal Projector}

To summarise, the embedding $\phi$ induces a mapping from vectors on $\Sigma$ to vectors on $M$ which we have defined as the push-forward mapping ($\phi_{*})$ additionally $\phi$ also induces a mapping between the 1-forms in $M$ to 1-forms in $\Sigma$ called the pull-back mapping ($\phi^{*}$). One cannot directly define a reverse map for either of these mappings for instance $F: T_{p}(M) &\to T_{p}(\Sigma)$ with coordinates $\vec{v}=(v^{t},v^{x},v^{y},v^{z}) &\mapsto F\vec{v}=(v^{x},v^{y},v^{z})$ cannot be reversely defined as it would depend on the choice of coordinates which is not the case in the push-forward mapping. 

If $\Sigma$ is a spacelike hypersurface, a coordinate independent reverse mapping is introduced by the \textit{orthogonal projector} onto $\Sigma$.

\begin{definition}(First fundamental form) The induced metric of our surface $\Sigma$ is the pull-back of the spacetime metric $\vec{g}$, which is a bilinear form, $$\vec{\gamma} := \phi^{*} \vec{g}$$

$\vec{\gamma}$ is an $D-1$-dimensional metric on $\Sigma$ called the \textit{the first fundamental form}

\end{definition}

$\gamma_{ij}$ can be viewed as a symmetric, linear operation at each point in $\Sigma$. Let $p \in \Sigma$, we have that $\vec{\gamma}:T_{p}\Sigma \times T_{p}\Sigma &\to \mathbb{R}$. We raise and lower the indices of vectors,tensors, etc. defined on $\Sigma$ using the spatial metric, i.e. the first fundamental form. Moreover in terms of the coordinate system, $x^{i} = (x,y,z)$ of $\Sigma$ the components of $\vec{\gamma}$ are deduced from \begin{center}

    \boxed{$$\tensor{\gamma}{_i_j} = \tensor{g}{_i_j}$$}

\end{center}

\textbf{\emph{Remark}} The hypersurface $\Sigma$ is said to be 

\begin{itemize}

    \item \textbf{spacelike} iff the metric $\vec{\gamma}$ is Riemannian: with signature $(-,+,+)$

    \item \textbf{timelike} iff the metric $\vec{\gamma}$ is Lorentzian: with signature $(+,+,+)$

    \item \textbf{null} iff the metric $\vec{\gamma}$ is degenerate: with signature $(0,+,+)$

\end{itemize}

Now we turn our attention to only hypersurfaces involved in the 3+1 formalism i.e. when $\Sigma$ is induced by the Riemannian metric \textit{$\vec{\gamma}$} or equivalently when the unit vector \textit{$\vec{n}$} normal to the hypersurface is timelike. 

\newline \begin{definition}(The Orthogonal Projector) is the operator $\vec{\overrightarrow{\gamma}}$ onto $\Sigma$ associated with the decomposition of the space of all spacetime vectors at each point $p \in \Sigma$: \newline \boxed{$$T_{p}(M) = T_{p}(\Sigma) \bigoplus span(\vec{n})$$} where span($\vec{n}$) is the 1-dimensional subspace of $T_p(M)$ created by vector $\vec{n}$. Therefore it is concluded that \begin{align}

  \vec{\overrightarrow{\gamma}}\colon T_{p}(M)&\to T_{p}(\Sigma)  \\

                                      \vec{v}&\mapsto \vec{v} + (\vec{n} \cdot \vec{v})\vec{n} \notag

\end{align}

\end{definition}


We have $\forall \vec{v} \in T_{p}(M), \;\vec{\overrightarrow{\gamma}}(\vec{v})=\vec{v}$ and consequently from Eq.(6.8) we get that $\vec{\overrightarrow{\gamma}}(\vec{n})=0$.

\newline

In result of Eq.(6.8), the components of $\vec{\overrightarrow{\gamma}}$ with respect to any basis of $T_{p}(M)$ are the following:

\begin{equation}

    \boxed{\tensor{\gamma}{^\mu_\alpha} = \tensor{\delta}{^\mu_\alpha}+n^{\mu}n_{\alpha}}

\end{equation}

As we can see from the very definition of the orthogonal projector, it is the mapping $T_{p}(M) &\to T_{p}(\Sigma)$ and so we can construct the reverse pull-back map $\vec{\overrightarrow{\gamma}}^{*}:T^{*}_{p}(\Sigma) &\to T^{*}_{p}(M)$ for any linear form $\underline{\vec{\omega}}$ in our $T^{*}_{p}(\Sigma)$. 

\begin{align*}

     \tensor{\vec{\overrightarrow{\gamma}}}{^*_M} \,\underline{\vec{\omega}}: T_{p}(M) &\to \mathbb{R} \\ 

    \vec{v} &\mapsto \langle \underline{\vec{\omega}}, \vec{\overrightarrow{\gamma}}(\vec{v}) \rangle

\end{align*}

Defining a linear form $\underline{\vec{\omega}}$ belonging to $T_{p}^{*}(M)$.

We now extend this operation to any multilinear form $\vec{A}$ acting on $T_{p}(\Sigma)$:

\begin{align}

    \tensor{\vec{\overrightarrow{\gamma}}}{^*_M} \, \,\vec{A}: T_{p}(M)^{*} &\to \mathbb{R} \notag \\ 

     (\vec{v_{1},v_{2},....,v_{n}})&\mapsto \vec{A}(\vec{\overrightarrow{\gamma}}(\vec{v_{1}})....\vec{\overrightarrow{\gamma}}(\vec{v_{n}}))

\end{align}

The way to think about the orthogonal projector operator which is defined on the 4D manifold is that when it acts on vectors or forms it gives back a vector or form which only has non-zero values on the \textit{spatial coordinates} i.e. $x^{i}$. Now we can extend this definition by pulling it back from 4D $M$ onto the 3D hypersurface where it acts as the induced metric $\vec{\gamma}$. So we shall apply the pull-back mapping of the orthogonal projector ($\tensor{\vec{\overrightarrow{\gamma}}}{^*_M}$) operator to a the bilinear form on $\Sigma$ which is constituted by the induced metric:$\boxed{\vec{\gamma} \defeq \tensor{\overrightarrow{\vec{\gamma}}}{^*_M}\vec{\gamma}}$

Now let's apply $\tensor{\overrightarrow{\vec{\gamma}}}{^*_M}$ to the extrinsic curvature $\vec{K}$ and extend its defintion to $M $ by: $$\boxed{\vec{K}\defeq \tensor{\overrightarrow{\vec{\gamma}}}{^*_M}\vec{K}}$$

In this paper we will be using the "4-dimensional point of view" approach, giving us a chance for easier manipulation of tensors on $\Sigma$ as if they are defined on $M$.

For covariant tensors (i.e. multilinear forms) the 4D extension is performed by the map $\overrightarrow{\vec{\gamma}}^{*}_{M}$ as seen above. The extension of 3-dimensional tensors to 4D ones is provided by using the pull-back operating on the \textit{orthogonal projector} for all tensors on $M$ in the following way:

\begin{align}

    \left( \overrightarrow{\vec{\gamma}}^{*} T \right) {^{\alpha_1}^ {\dotsb} ^{\alpha_p}_{\beta_1}_{\dotsb}_{\beta_q}} = \tensor{\gamma}{^{\alpha_1}_{\mu_1}} \dotsb \tensor{\gamma}{^{\alpha_p}_{\mu_q}} \tensor{\gamma}{^{\nu_1}_{\beta_1}} \dotsb \tensor{\gamma}{^{\nu_p}_{\beta_q}} \tensor{T}{^{\mu_1} ^{\dotsb}^{\mu_p} _{\nu_1} _{\dotsb} _{\nu_q}  }

\end{align}

\textbf{Remarks:} \begin{itemize} \item For a multilinear form $A$ on $\Sigma$ we have the following $\overrightarrow{\vec{\gamma}}^{*}(\tensor{\overrightarrow{\vec{\gamma}}}{^*_M} A) = \overrightarrow{\vec{\gamma}}^{*}_{M} (A)$ \item $\forall \vec{v} \in T_{p}(M) \colon \overrightarrow{\vec{\gamma}}^{*}\vec{v} = \overrightarrow{\vec{\gamma}}(\vec{v})$

\item For a linear form $\underline{\vec{\omega}} \in \tensor{T}{^*_p}(M), \overrightarrow{\vec{\gamma}}^{*} \underline{\vec{\omega}} = \underline{\vec{\omega}} \circ \overrightarrow{\vec{\gamma}}$ 

\item Finally we have that for any tensor $T$, $\overrightarrow{\vec{\gamma}}^{*}T$ is tangent to $\Sigma$.

\end{itemize}

\textbf{Note:} The relation between the spacetime metric and the 3-metric($\vec{\gamma}$) is given by:

$$\boxed{\tensor{\gamma}{_\alpha_\beta}=\tensor{g}{_\alpha_\beta}+n_{\alpha} n_{\beta}}.$$

\subsection{Relation Between $K$ and $\nabla_{\vec{n}}$}

The unit vector $\vec{n}$ is normal to $\Sigma$ by only points on $\Sigma$. Now if we consider some extension of $\vec{n}$ in the neighbourhood around the hypersurface. The natural extension can be provided by the gradient of the scalar field $t$ on $\Sigma$. [cf. Eq.(6.2)]

\newline Recall $\vec{a} \defeq \nabla_{\vec{n}}\vec{n}$, where $\vec{n}$ is considered as the 4-velocity and $\vec{a}$ the corresponding 4-acceleration of some Eulerian observer. Let us next explicitly define the tensor $\vec{K}$ extended to $M$ ($\vec{K}\defeq \tensor{\overrightarrow{\vec{\gamma}}}{^*_M}\vec{K}$) using the definition of the operator $\tensor{\overrightarrow{\vec{\gamma}}}{^*_M}$ Eq.(6.10) and $\vec{K}$, Eq.(6.7). Additionally, keep in mind the definition of the Orthogonal Projector defined in Eq. (6.8):

 

\begin{align}

    \forall (\vec{u},\vec{v}) \in T_{p}(M)^{2},  \; \; \vec{K}(\vec{u},\vec{v}) &= \vec{K}(\overrightarrow{\vec{\gamma}}(\vec{u}),\overrightarrow{\vec{\gamma}}(\vec{v})) = - \overrightarrow{\vec{\gamma}}(\vec{u}) \cdot \nabla_{\overrightarrow{\vec{\gamma}}(\vec{v})}\vec{n} \notag \\ 

    &= - \overrightarrow{\vec{\gamma}}(\vec{u}) \cdot \nabla_{\vec{v}+(\vec{n} \cdot \vec{v})\vec{n}}\vec{n}\notag \\

    &= -[\vec{u}+(\vec{n} \cdot \vec{u})\vec{n}]\cdot [\nabla_{\vec{v}}\vec{n}+(\vec{n}\cdot \vec{v})\nabla_{\vec{n}}\vec{n}] \notag \\ &= -\vec{u} \cdot \nabla_{\vec{v}}\vec{n} - (\vec{n}\cdot \vec{v})\vec{u} \cdot \underbrace{\nabla_{\vec{n}}\vec{n}}_{= \vec{a}} - (\vec{n} \cdot \vec{u})\underbrace{\vec{n} \cdot \nabla_{\vec{v}}\vec{n}}_{= 0} - (\vec{n}\cdot \vec{u})(\vec{n} \cdot \vec{v})\underbrace{\vec{n} \cdot \nabla_{\vec{n}}\vec{n}}_{= 0} \notag

     \\ &= -\vec{u} \cdot \nabla_{\vec{v}}\vec{n} - (\vec{a} \cdot \vec{u})(\vec{n} \cdot \vec{v})\notag \\ 

     &= - \nabla{\underline{\vec{n}}}(\vec{u},\vec{v}) - \langle \underline{\vec{a}},\vec{u} \rangle \langle \underline{\vec{n}}, \vec{v} \rangle.

\end{align}

Where we refer to Eq. (6.3) to deduce $\vec{n} \cdot \nabla_{\vec{x}}\vec{n} = 0$ for any vector $\vec{x}$. Note that in Eq. (6.12) $\underline{\vec{a}}$ and $\underline{\vec{n}}$ are 1-forms. 

Since Eq. (6.12) is valid for any pair of vectors $(\vec{u},\vec{v})$ we can conclude the following: \begin{equation} \boxed{\nabla\vec{\underline{n}} = -\vec{K} - \vec{\underline{a}} \bigotimes \vec{\underline{n}}}   \end{equation} 

In components: \begin{equation*}

    \nabla_{\beta}n_{\alpha} = -\tensor{K}{_\alpha_\beta} - a_{\alpha}n_{\beta}

\end{equation*}

You could say that Eq.(6.13) shows that the extended extrinsic curvature on $M$ is the gradient of the 1-form $\underline{\vec{n}}$ to which the projector operator is applied to:

\begin{equation}

   \boxed{ \vec{K} = - \tensor{\overrightarrow{\vec{\gamma}}}{^*_M}\nabla\vec{\underline{n}}}

\end{equation}

\subsection{Links between the $\nabla$ and $\vec{D}$ connection}

The Levi-Civita connection $(\vec{D})$ in ${\vec{\gamma},\Sigma}$ on a given tensor field $\vec{T}$ can also be expressed in terms of the covariant derivatice ($\nabla$) according to the following formula: \begin{equation}

    \boxed{\vec{D}\vec{T} \defeq \overrightarrow{\vec{\gamma}}^{*} \nabla \vec{T}}

\end{equation}

in components:

\begin{align*}

    D_{\rho}\tensor{T}{\;^{\alpha_1}^\dots^{\alpha_p}\,_{\beta_1}_\dots_{\beta_q}} = \tensor{\gamma}{^{\alpha_1}_{\mu_1}} \dots \tensor{\gamma}{^{\alpha_p}_{\mu_1}}\tensor{\gamma}{^{\nu_1}_{\beta_1}} \dots \tensor{\gamma}{^{\nu_q}_{\beta_q}}\tensor{\gamma}{^\sigma_\rho}\nabla_{\sigma}\tensor{T}{\;^{\mu_1}^\dots^{\mu_p}\,_{\nu_1}_\dots_{\nu_q}}

\end{align*}

As you have probably guessed it, $\vec{T}$ on the right hand side of Eq.(6.15) is the 4-dimensional extension provided by Eq.(6.10), i.e. $\vec{T} \defeq \overrightarrow{\vec{\gamma}}^{*}_{M} \vec{T}$ as the connection $\nabla$ acts on tensors on our 4D manifold $M$. Furthermore, the right hand side should write $\overrightarrow{\vec{\gamma}}^{*}_{M} \vec{D}\vec{T}$ for equality purposes between tensors on $M$. Therefore we can re-write formula (6.16) in a more detailed way: \begin{align}

    \overrightarrow{\vec{\gamma}}^{*}_{M} \vec{D}\vec{T} = \overrightarrow{\vec{\gamma}}^{*}[\nabla(\overrightarrow{\vec{\gamma}}^{*}_{M}\vec{T})]

\end{align}

In our four-dimensional tensor $\vec{T}$, it's domain of definition is the hypersurface $\Sigma$. Hence if one would like to define $\nabla\vec{T}$, then initially the domain of definition should be expanded to an open set of $M$. In order to do so, one makes some extension $\vec{T'}$ of $\vec{T}$ which is an open neighbourhood of $\Sigma$ in $M$ and then compute $\nabla\vec{T'}$ accordingly. The choice of extension here is irrelevant thanks to the operator $\overrightarrow{\vec{\gamma}}^{*}$ acting on $\nabla\vec{T'}$ providing that $\vec{T'}=\vec{T}$ at every point in $\Sigma$. We will demonstrate (6.15) in two steps: \begin{enumerate}

    \item Showing that the pull-back of the connection ($\overrightarrow{\vec{\gamma}}^{*}\nabla$) or more precisely, $\overrightarrow{\vec{\gamma}}^{*}\nabla\overrightarrow{\vec{\gamma}}^{*}_{M}$, is torsion-free.

    \item The connection vanishes when applied to the metric tensor $\vec{\gamma}$.

\end{enumerate}

In our calculation we will make use of Eq.(6.11) and the fact that the extended metric $\vec{\gamma}$ can be expressed in terms of the four-dimensional metric tensor $\vec{g}$ and 1-form $\underline{\vec{n}}$ in the following manner: 

\begin{align*}

    \tensor{(\overrightarrow{\vec{\gamma}}^{*}\nabla\vec{\gamma})}{_\alpha_\beta_\phi} &= \tensor{\gamma}{^\mu_\alpha}\tensor{\gamma}{^\nu_\beta}\tensor{\gamma}{^\rho_\phi} \nabla_{\rho}\tensor{\gamma}{_\mu_\nu} \\

    &= \tensor{\gamma}{^\mu_\alpha}\tensor{\gamma}{^\nu_\beta}\tensor{\gamma}{^\rho_\phi} \nabla_{\rho}(\tensor{g}{_\mu_\nu} + \tensor{n}{_\mu}\tensor{n}{_\nu}) \\ &= \tensor{\gamma}{^\mu_\alpha}\tensor{\gamma}{^\nu_\beta}\tensor{\gamma}{^\rho_\phi}[\underbrace{\nabla_{\rho}\tensor{g}{_\mu_\nu}}_{= 0}+(\nabla_{\rho}n_{\mu})n_{\nu} + n_{\mu}(\nabla_{\rho}n_{\nu})] \\ &= \tensor{\gamma}{^\rho_\phi} \left( \tensor{\gamma}{^\mu_\alpha} \underbrace{\tensor{\gamma}{^\nu_\beta}n_{\nu}}_{= 0} (\nabla_{\rho}n_{\mu}) + \underbrace{\tensor{\gamma}{^\mu_\alpha}n_{\mu}}_{= 0} \tensor{\gamma}{^\nu_\beta} (\nabla_{\rho}n_{\nu}) \right) \\ &=  0 

\end{align*}

Due to the uniqueness of the torsion-free Levi Civita connection associated with the non-degenrate metric $\vec{\gamma}$, we can conclude that $\overrightarrow{\vec{\gamma}}^{*}\nabla=\vec{D}$.

Next we'll show the equivalent of Eq.(6.15) on a given vector field $\vec{v}$ along another vector field $\vec{u}$ with both vectors tangent to $\Sigma$. Certainly we get:

\begin{align*}

    (D_{\vec{u}})v^{\alpha} &= u^{\lambda}D_{\lambda}v^{\alpha} = \underbrace{u^{\lambda}\tensor{\gamma}{^\nu_\lambda}}_{= u^{\nu}}\tensor{\gamma}{^\alpha_\mu}\nabla_{\nu}v^{\mu} \\ &= u^{\nu}(\tensor{\delta}{^\alpha_\mu}+n^{\alpha}n_{\mu})\nabla_{\nu}v^{\mu} \\ &= u^{\nu}\nabla_{\nu}\tensor{\delta}{^\alpha_\mu}v^{\mu}+ n^{\alpha}u^{\nu}\underbrace{n_{\mu}\nabla_{\nu}v^{\mu}}_{-v^{\mu}\nabla_{\nu}n_{\mu}} \\ &= u^{\nu}\nabla_{\nu}v^{\alpha} - n^{\alpha}u^{\nu}v^{\mu}\nabla_{\nu}n_{\mu}

\end{align*}

In the last line we have used the fact that when $\vec{v}$ is tangent to $\Sigma$ we have $\vec{n}\cdot\vec{v} = 0$ and so $n_{\mu}\nabla_{\nu}v^{\mu}=-v^{\mu}\nabla_{\nu}n_{\mu}$ which is obvious by taking the covariant derivative of $(\vec{n} \cdot \vec{v}) = 0$ and hence getting $n_{\mu}\nabla_{\nu}v^{\mu} + v^{\mu}\nabla_{\nu}n_{\mu} = 0$.

\newline From definition (6.7) we can re-write $u^{\nu}v^{\mu}\nabla_{\nu}n_{\mu} = -\vec{K}(\vec{v},\vec{u}) =-\vec{K}(\vec{u},\vec{v}) $ since $\vec{K}$ is symmetric. Therefore the above finally becomes

\begin{align}

    \forall (\vec{u},\vec{v}) \in T(\Sigma) \times T(\Sigma), \; \;

    \boxed{\vec{D}_{\vec{u}}\vec{v} = \nabla_{\vec{u}}\vec{v} + \vec{K}(\vec{u},\vec{v})\vec{n}}

\end{align}

You could say this final result grants us another interpretation for the extrinsic curvature $\vec{K}$; by taking the intrinsic curvature $\vec{D}$ of $\Sigma$ from the derivative taken with the spacetime connection $\nabla$, $\vec{K}$ measures the deviation of any vector of $\Sigma$ along another vector on the hypersurface.

\subsection{Gauss-Codazzi relations}

In this last section of chapter 6 we will discover the core of the 3+1 formalism of general relativity by showing the decomposition of the Riemann curvature tensor on hypersurface $\Sigma$, associated with the induced metric $\vec{\gamma}$ and the extrinsic curvature tensor of $\Sigma$, $\vec{K}$.

\subsubsection{Gauss Relation}

We will first start with the 4-dimensional version of the Ricci identity in Eq.(6.4) on a generic vector field tangent to $\Sigma$:

\begin{equation}

    D_{\alpha}D_{\beta}v^{\lambda} - D_{\beta}D_{\alpha}v^{\lambda} = \tensor{R}{^\lambda_\mu_\alpha_\beta}v^{\mu}

    \end{equation}

Now we make use of the component version of formula (6.15) to convert the above from the $\vec{D}$-derivative to the $\nabla$-derivative:

\[

D_{\alpha}D_{\beta}v^{\lambda} = D_{\alpha}(D_{\beta}v^{\lambda}) = \tensor{\gamma}{^\mu_\alpha}\tensor{\gamma}{^\nu_\beta}\tensor{\gamma}{^\lambda_\rho}\nabla_{\mu}(D_{\nu}v^{\rho})

\]

and now applying the same principle to derive $D_{\nu}v^{\rho}$:

\[

D_{\alpha}D_{\beta}v^{\lambda}  = \tensor{\gamma}{^\mu_\alpha}\tensor{\gamma}{^\nu_\beta}\tensor{\gamma}{^\lambda_\rho}\nabla_{\mu}(D_{\nu}v^{\rho}) = \tensor{\gamma}{^\mu_\alpha}\tensor{\gamma}{^\nu_\beta}\tensor{\gamma}{^\lambda_\rho}\nabla_{\mu}(\tensor{\gamma}{^\sigma_\nu}\tensor{\gamma}{^\rho_\gamma}\nabla_{\sigma}v^{\gamma})

\]

We shall now use Eq.(6.11) to replace $\nabla_{\mu}(\tensor{\gamma}{^\sigma_\nu})$ by $(\nabla_{\mu}n^{\sigma})n_{\nu} + n^{\sigma}(\nabla_{\mu}n_{\nu})$ with the help of the Leibniz rule and the fact that $\nabla_{\mu}\tensor{\delta}{^\sigma_\nu}=0$:

\begin{align*}

   \tensor{\gamma}{^\mu_\alpha}\tensor{\gamma}{^\nu_\beta}\tensor{\gamma}{^\lambda_\rho}\nabla_{\mu}(\tensor{\gamma}{^\sigma_\nu}\tensor{\gamma}{^\rho_\gamma}\nabla_{\sigma}v^{\gamma}) &= \tensor{\gamma}{^\mu_\alpha}\tensor{\gamma}{^\nu_\beta}\tensor{\gamma}{^\lambda_\rho}[(\nabla_{\mu}n^{\sigma})n_{\nu} + n^{\sigma}(\nabla_{\mu}n_{\nu})]\tensor{\gamma}{^\rho_\gamma}\nabla_{\sigma}v^{\gamma} \\

   &= \tensor{\gamma}{^\mu_\alpha}\tensor{\gamma}{^\nu_\beta}\tensor{\gamma}{^\lambda_\rho}[n^{\sigma}(\nabla_{\mu}n_{\nu})\tensor{\gamma}{^\rho_\gamma}\nabla_{\sigma}v^{\gamma} + n_{\nu}(\lambda_{\mu}n^{\sigma})\tensor{\gamma}{^\rho_\gamma}\nabla_{\sigma}v^{\gamma}] \\

   &= \tensor{\gamma}{^\mu_\alpha}\tensor{\gamma}{^\nu_\beta}\tensor{\gamma}{^\lambda_\rho} \left(n^{\sigma}\nabla_{\mu}n_{\nu}\tensor{\gamma}{^\rho_\gamma}\nabla_{\sigma}v^{\gamma}\right)   

\end{align*} 

where $\tensor{\gamma}{^\nu_\beta}n_{\nu} = 0$.

\newline 

\smallskip

We apply the same principle as the above for $\nabla_{\mu}\tensor{\gamma}{^\rho_\gamma}$ to get the following:

\begin{align}

D_{\alpha}D_{\beta}v^{\lambda} &= \tensor{\gamma}{^\mu_\alpha}\tensor{\gamma}{^\nu_\beta}\tensor{\gamma}{^\lambda_\rho} \Big(n^{\sigma}\nabla_{\mu}n_{\nu}\tensor{\gamma}{^\rho_\gamma}\nabla_{\sigma}v^{\gamma} + \tensor{\gamma}{^\sigma_\nu}\nabla_{\mu}n^{\rho}\underbrace{n_{\gamma}\nabla_{\sigma}v^{\gamma}}_{= -v^{\gamma}\nabla_{\sigma}n_{\gamma}} + \tensor{\gamma}{^\sigma_\nu}\tensor{\gamma}{^\rho_\gamma}\nabla_{\mu}\nabla_{\sigma}v^{\gamma}\Big) \notag\\

&= \tensor{\gamma}{^\mu_\alpha}\tensor{\gamma}{^\nu_\beta}\tensor{\gamma}{^\lambda_\gamma}\nabla_{\mu}n_{\nu}n^{\sigma}\nabla_{\sigma}v^{\gamma}-\tensor{\gamma}{^\mu_\alpha}\tensor{\gamma}{^\sigma_\beta}\tensor{\gamma}{^\lambda_\rho}v^{\gamma}\nabla_{\mu}n^{\rho}\nabla_{\sigma}n_{\gamma}+\tensor{\gamma}{^\mu_\alpha}\tensor{\gamma}{^\sigma_\beta}\tensor{\gamma}{^\lambda_\gamma}\nabla_{\mu}\nabla_{\sigma}v^{\gamma} \notag \\

&= -\tensor{K}{_\alpha_\beta}\tensor{\gamma}{^\lambda_\gamma}n^{\sigma}\nabla_{\sigma}v^{\gamma}- \tensor{K}{^\lambda_\alpha}\tensor{K}{_\beta_\gamma}v^{\gamma}+\tensor{\gamma}{^\mu_\alpha}\tensor{\gamma}{^\sigma_\beta}\tensor{\gamma}{^\lambda_\gamma}\nabla_{\mu}\nabla_{\sigma}v^{\gamma}

\end{align}

where we used the idempotence of the projection operation $\overrightarrow{\gamma}$, i.e. $\tensor{\gamma}{^\lambda_\rho}\tensor{\gamma}{^\rho_\gamma}=\tensor{\gamma}{^\lambda_\gamma}$ and have referred to Eq.(6.14) for $\tensor{\gamma}{^\mu_\alpha}\tensor{\gamma}{^\nu_\beta}\nabla_{\mu}n_{\nu} = - \tensor{K}{_\beta_\alpha}$ to get the second and third lines. Now going back to our Ricci identity in Eq.(6.18), we premute the $\alpha$ and $\beta$ indices and subtract from Eq.(6.19) to derive $D_{\alpha}D_{\beta}v^{\lambda}-D_{\beta}D_{\alpha}v^{\lambda}$. So according to Eq.(6.19),

$D_{\beta}D_{\alpha}v^{\lambda} = -\tensor{K}{_\beta_\alpha}\tensor{\gamma}{^\lambda_\gamma}n^{\sigma}\nabla_{\sigma}v^{\gamma} - \tensor{K}{^\lambda_\beta}\tensor{K}{_\alpha_\gamma}v^{\gamma}+\tensor{\gamma}{^\sigma_\alpha}\tensor{\gamma}{^\mu_\beta}\tensor{\gamma}{^\lambda_\gamma}\nabla_{\sigma}\nabla_{\mu}v^{\gamma}$. We therefore get:

\begin{equation*}

  D_{\alpha}D_{\beta}v^{\lambda}-D_{\beta}D_{\alpha}v^{\lambda} = \Big(\tensor{K}{_\alpha_\mu}\tensor{K}{^\lambda_\beta} - \tensor{K}{_\beta_\mu}\tensor{K}{^\lambda_\alpha} \Big)v^{\mu} + \tensor{\gamma}{^\rho_\alpha}\tensor{\gamma}{^\sigma_\beta}\tensor{\gamma}{^\lambda_\gamma}\Big(\nabla_{\rho}\nabla_{\sigma}v^{\gamma}-\nabla_{\sigma}\nabla_{\rho}v^{\gamma}\Big)

\end{equation*}

The first term vanishes since $\tensor{K}{_\alpha_\beta}$ is symmetric.

We apply the Ricci identity to for connection $\nabla$ to get the 4-dimensional Riemann curvature tensor: $$\nabla_{\rho}\nabla_{\sigma}v^{\gamma}-\nabla_{\sigma}\nabla_{\rho}v^{\gamma}= \; ^{4}\tensor{R}{^\gamma_\mu_\rho_\sigma}v^{\mu}$$

Hence the above become 

\begin{equation*}

     D_{\alpha}D_{\beta}v^{\lambda}-D_{\beta}D_{\alpha}v^{\lambda} = \Big(\tensor{K}{_\alpha_\mu}\tensor{K}{^\lambda_\beta} - \tensor{K}{_\beta_\mu}\tensor{K}{^\lambda_\alpha} \Big)v^{\mu} + \tensor{\gamma}{^\rho_\alpha}\tensor{\gamma}{^\sigma_\beta}\tensor{\gamma}{^\lambda_\gamma}\tensor{R}{^\gamma_\mu_\rho_\sigma}v^{\mu}

\end{equation*}

Substituting this relation for the left-hand side of Eq.(6.18) we get

\begin{equation*}

  \Big(\tensor{K}{_\alpha_\mu}\tensor{K}{^\lambda_\beta} - \tensor{K}{_\beta_\mu}\tensor{K}{^\lambda_\alpha} \Big)v^{\mu} + \tensor{\gamma}{^\rho_\alpha}\tensor{\gamma}{^\sigma_\beta}\tensor{\gamma}{^\lambda_\gamma}^{4}\tensor{R}{^\gamma_\mu_\rho_\sigma}v^{\mu} = \tensor{R}{^\lambda_\mu_\alpha_\beta}v^{\mu} 

\end{equation*}

We then set $v^{\mu}=\tensor{\gamma}{^\mu_\nu}v^{\nu}$,

\begin{equation*}

  \tensor{\gamma}{^\mu_\alpha}\tensor{\gamma}{^\nu_\beta}\tensor{\gamma}{^\lambda_\rho}\tensor{\gamma}{^\sigma_\gamma} ^{4}\tensor{R}{^\rho_\sigma_\mu_\nu}v^{\gamma} = \tensor{R}{^\lambda_\gamma_\alpha_\beta}v^{\gamma} + \Big( \tensor{K}{^\lambda_\alpha}\tensor{K}{_\gamma_\beta} - \tensor{K}{^\lambda_\beta}\tensor{K}{_\alpha_\gamma} \Big)v^{\gamma}.  

\end{equation*}

In the above identity, $\vec{v}$ can be any vector in the tangent space $T(M)$ without manipulating the results. Thanks to the  projector operator and the fact that both $\vec{K}$ and the Riemann curvature tensor on $\Sigma$ (without superscript 4) are tangent to $\Sigma$ we conclude:

\begin{equation}

    \boxed{\tensor{\gamma}{^\mu_\alpha}\tensor{\gamma}{^\nu_\beta}\tensor{\gamma}{^\lambda_\rho}\tensor{\gamma}{^\sigma_\delta} ^{4}\tensor{R}{^\rho_\sigma_\mu_\nu} = \tensor{R}{^\lambda_\delta_\alpha_\beta}+\tensor{K}{^\lambda_\alpha}\tensor{K}{_\delta_\beta}-\tensor{K}{^\lambda_\beta}\tensor{K}{_\alpha_\delta}}

\end{equation}

This is the \textbf{\textit{Gauss relation.}}

\newline

If we contract the Gauss relation on the indices $\alpha$ and $\lambda$ and have $\tensor{\gamma}{^\mu_\alpha}\tensor{\gamma}{^\alpha_\rho}=\tensor{\gamma}{^\mu_\rho}=\tensor{\delta}{^\mu_\rho}+n^{\mu}n_{\rho}$, the expression obtained has the Ricci tensors $^{4}\vec{R}$ which is associated with $(M,\vec{g})$ and $\vec{R}$ on $(\Sigma,\vec{\gamma})$:

\begin{equation}

  \boxed{\tensor{\gamma}{^\mu_\alpha}\tensor{\gamma}{^\nu_\beta}^{4}\tensor{R}{_\mu_\nu} + \tensor{\gamma}{_\alpha_\mu}n^{\nu}\tensor{\gamma}{^\rho_\beta}n^{\sigma}^{4}\tensor{R}{^\mu_\nu_\rho_\sigma} = \tensor{R}{_\alpha_\beta} + K\tensor{K}{_\alpha_\beta}-\tensor{K}{_\alpha_\mu}\tensor{K}{^\mu_\beta}}

\end{equation}

The above is instinctively called the \textbf{\textit{contracted Gauss relation.}}\newline The left-hand side of Eq.(6.21) has been derived in the following manner (keeping our free indices as $\alpha$ and $\beta$ and making use of the Riemann tensor symmetries):

\begin{align*}

    \tensor{\gamma}{^\mu_\alpha}\tensor{\gamma}{^\lambda_\rho} \tensor{\gamma}{^\nu_\beta}\tensor{\gamma}{^\sigma_\delta} ^{4}\tensor{R}{^\rho_\sigma_\mu_\nu} &=\tensor{\gamma}{^\mu_\rho}\tensor{\gamma}{^\nu_\beta}\tensor{\gamma}{^\sigma_\alpha}^{4}\tensor{R}{^\rho_\sigma_\mu_\nu} \\ &= \big(\tensor{\delta}{^\mu_\rho} + n^{\mu}n_{\rho} \big) \tensor{\gamma}{^\nu_\beta}\tensor{\gamma}{^\sigma_\alpha} ^{4}\tensor{R}{^\rho_\sigma_\mu_\nu} \\ &= \tensor{\gamma}{^\mu_\alpha} \tensor{\gamma}{^\nu_\beta} ^{4}\tensor{R}{_\mu_\nu} + n^{\mu} n^{\rho} \tensor{\gamma}{^\nu_\beta} \tensor{\gamma}{^\sigma_\alpha} ^{4}\tensor{R}{_\rho_\sigma_\mu_\nu} \\ &= \tensor{\gamma}{^\mu_\nu} \tensor{\gamma}{^\nu_\beta} ^{4}\tensor{R}{_\mu_\alpha} + n^{\mu} n^{\rho} \tensor{\gamma}{^\nu_\beta} \tensor{\gamma}{^\sigma_\alpha} ^{4}\tensor{R}{_\mu_\nu_\rho_\sigma} \intertext{after playing around with dummy indices in the second term one gets:} &= \tensor{\gamma}{^\mu_\alpha} \tensor{\gamma}{^\nu_\beta} ^{4}\tensor{R}{_\mu_\nu} + \tensor{\gamma}{_\alpha_\mu} n^{\nu} \tensor{\gamma}{^\rho_\beta} n^{\sigma} ^{4}\tensor{R}{^\mu_\nu_\rho_\sigma}

\end{align*} 

\newline Our next task is to derive the \textbf{\textit{scalar Gauss relation}} which we will obtain by taking the trace of Eq.(6.21) and contracting it with $\vec{\gamma}$:

\begin{equation}

    \boxed{^{4}R + 2 ^{4}\tensor{R}{_\mu_\nu}n^\mu n^\nu = R + K^{2} - \tensor{K}{_i_j}\tensor{K}{^i^j}}

\end{equation}

Note that $\tensor{K}{^\mu_\mu} = \tensor{K}{^i_i} = K, \; \tensor{K}{_\mu_\nu}\tensor{K}{^\mu^\nu}=\tensor{K}{_i_j}\tensor{K}{^i^j}$ and we also have that:

$$\tensor{\gamma}{^\alpha^\beta}\tensor{\gamma}{_\alpha_\mu}n^{\nu}\tensor{\gamma}{^\rho_\beta}n^{\sigma}^{4}\tensor{R}{^\mu_\nu_\rho_\sigma}= \tensor{\gamma}{^\rho_\mu}n^{\nu}n^{\sigma} \;^{4}\tensor{R}{^\mu_\nu_\rho_\sigma} = \; ^{4}\tensor{R}{^\mu_\nu_\mu_\sigma}n^{\nu}n^{\sigma}+ \; ^{4}\tensor{R}{^\mu_\nu_\rho_\sigma}n^{\rho}n_{\mu}n^{\nu}n^{\sigma} = \;  ^{4}\tensor{R}{_\mu_\nu}n^{\mu}n^{\nu}$$

Eq.(6.22) gives rise to Gauss' \textit{remarkable theorem} (\textbf{\textit{Theorema Engregium}}) which is the relation between the curvature of spacetime, represented by the Ricci scalar $R$, and $\Sigma$'s extrinsic curvature represented by $K^{2}-\tensor{K}{_i_j}\tensor{K}{^i^j}$. The ignition to this remarkable theorem was when Gauss tried to figure out how two surfaces can be obtained from one another by \textit{bending without stretching}? So he came up with an ingenious way of measuring curvatures of curves on a surface. 

\subsubsection{Codazzi Relation}

We revisit the Ricci identity in Eq.(3.1) however applying it to the extension of the normal vector $\vec{n}$ in a neighbourhood of $\Sigma$ instead:

\begin{equation}

    \big( \nabla_{\alpha}\nabla_{\beta} - \nabla_{\beta}\nabla_{\alpha} \big)n^{\lambda} = \; ^{4}\tensor{R}{^\lambda_\mu_\alpha_\beta}n^{\mu}

\end{equation}

We project this relation onto $\Sigma$ and get

\begin{equation}

    \tensor{\gamma}{^\mu_\alpha}\tensor{\gamma}{^\nu_\beta}\tensor{\gamma}{^\lambda_\rho} ^{4}\tensor{R}{^\rho_\sigma_\mu_\nu}n^{\sigma} = \tensor{\gamma}{^\mu_\alpha}\tensor{\gamma}{^\nu_\beta}\tensor{\gamma}{^\lambda_\rho} \big( \nabla_{\mu}\nabla_{\nu}n^{\rho} - \nabla_{\nu}\nabla_{\mu}n^{\rho} \begi)

\end{equation}

With the help of Eq.(6.12) we derive

\begin{align}

  \tensor{\gamma}{^\mu_\alpha}\tensor{\gamma}{^\nu_\beta}\tensor{\gamma}{^\lambda_\rho}\nabla_{\mu}\nabla_{\nu}n^{\rho} &= \tensor{\gamma}{^\mu_\alpha}\tensor{\gamma}{^\nu_\beta}\tensor{\gamma}{^\lambda_\rho}\nabla_{\mu}\big(-\tensor{K}{^\rho_\nu} - a^{\rho}n_{\nu} \big) \notag \\

  &= - \tensor{\gamma}{^\mu_\alpha}\tensor{\gamma}{^\nu_\beta}\tensor{\gamma}{^\lambda_\rho} \big(\nabla_{\mu}\tensor{K}{^\rho_\nu} + \nabla_{\mu}a^{\rho}n_{\nu}+a^{\rho}\nabla_{\mu}n_{\nu} \big) \notag\\

  &= -\tensor{\gamma}{^\mu_\alpha}\tensor{\gamma}{^\nu_\beta}\tensor{\gamma}{^\lambda_\rho}\nabla_{\mu}\tensor{K}{^\rho_\nu} -\tensor{\gamma}{^\mu_\alpha}\underbrace{\tensor{\gamma}{^\nu_\beta}n_{\nu}}_{= 0}\tensor{\gamma}{^\lambda_\rho}\nabla_{\mu}a^{\rho} - \underbrace{\tensor{\gamma}{^\mu_\alpha}\tensor{\gamma}{^\nu_\beta}\nabla_{\mu}n_{\nu}}_{= -\tensor{K}{_\alpha_\beta}}\underbrace{\tensor{\gamma}{^\lambda_\rho}a^{\rho}}_{= a^{\lambda}} \notag \\ 

  &= -D_{\alpha}\tensor{K}{^\lambda_\beta} + a^{\lambda}\tensor{K}{_\alpha_\beta}

\end{align}

We have used Eq.(6.14) in order to get the first term in the last line. Hence we can agree that after permuting the indices $\alpha$ and $\beta$, we can accordingly derive :$$\tensor{\gamma}{^\mu_\alpha}\tensor{\gamma}{^\nu_\beta}\tensor{\gamma}{^\lambda_\rho}\nabla_{\nu}\nabla_{\mu}n^{\rho} = -D_{\beta}\tensor{K}{^\lambda_\alpha}+a^{\lambda}\tensor{K}{_\beta_\alpha}$$ Subtracting the above from Eq.(6.25), keeping in mind that $\vec{K}$ is symmetric, we see that Eq.(6.24) becomes

\begin{equation}

    \boxed{\tensor{\gamma}{^\lambda_\rho}n^{\sigma}\tensor{\gamma}{^\mu_\alpha}\tensor{\gamma}{^\nu_\beta}^{4}\tensor{R}{^\rho_\sigma_\mu_\nu} = D_{\beta}\tensor{K}{^\lambda_\alpha}-D_{\alpha}\tensor{K}{^\lambda_\beta}}

\end{equation}

This is the \textbf{\textit{Codazzi relation}} also known as the \textbf{\textit{Codazzi-Mainardi relation.}} \newline A further identity which arises from considering projections of the Riemann curvature tensor along the normal direction, involving a spatial derivative of the extrinsic curvature.

\newline Next we contract the Codazzi relation on the indices $\alpha$ and $\lambda$ to get:

$$\tensor{\gamma}{^\mu_\rho}n^{\sigma}\tensor{\gamma}{^\nu_\beta} ^{4}\tensor{R}{^\rho_\sigma_\mu_\nu} = D_{\beta}K - D_{\mu}\tensor{K}{^\mu_\beta}$$

Now breaking down the left-hand side one gets:

\begin{align*}

    \tensor{\gamma}{^\mu_\rho}n^{\sigma}\tensor{\gamma}{^\nu_\beta} ^{4}\tensor{R}{^\rho_\sigma_\mu_\nu} &= (\tensor{\delta}{^\mu_\rho}+n^{\mu}n_{\rho})n^{\sigma}\tensor{\gamma}{^\nu_\beta}^{4}\tensor{R}{^\rho_\sigma_\mu_\nu} \\

    &= n^{\sigma}\tensor{\gamma}{^\nu_\beta}\tensor{R}{_\sigma_\nu} + \tensor{\gamma}{^\nu_\beta}^{4}\tensor{R}{^\rho_\sigma_\mu_\nu}n_{\rho}n^{\sigma}n^{\mu}

\end{align*}

In the second term, due to the antisymmetry of the first two indices, the Riemann tensor vanishes. Therefore we are left with

\begin{equation}

    \boxed{\tensor{\gamma}{^\mu_\alpha}n^{\nu}^{4}\tensor{R}{_\mu_\nu}=D_{\alpha}K-D_{\mu}\tensor{K}{^\mu_\alpha}}

\end{equation}

This equation is called the \textbf{\textit{contracted Codazzi relation.}}

\section{Geometry of Foliations and the 3+1 Formalism}

So far we have mostly focused on the geometry of hypersurfaces where most results have been independent of the Einstein equation and the choice of coordinates $(x^{i})$ on $\Sigma$.

\newline Next we'll introduce ourselves to the 3+1 formalism of general relativity and its foundation which is the \textbf{\textit{foliation}} of spacetime.

\subsection{Globally Hyperbolic spacetimes}

A spacelike hypersurface $\Sigma$ in $M$ is called a \textit{Cauchy surface} if every worldline (i.e. timelike curve) without endpoints intersects $\Sigma$ once and only once. A spacetime $(M,\vec{g})$ with a Cauchy surface is said to be \textbf{\textit{globally hyperbolic}}. Note that not all spacetimes admit a Cauchy surface.

\subsection{Definition of Foliation}

A globally hyperbolic spacetime $(M,\vec{g})$ can be \textbf{\textit{foliated}} (or sliced) into stacks of spacelike hypersurfaces where each $\Sigma_{t}$ is separated by a time-step $t$. Our time coordinate $t$ is said to be "regular" due to having non-vanishing gradient

and therefore is used as a label for time slices. Consider a smooth scalar field $\hat{t}$ on $M$, we have the following: $$\forall t \in \mathbb{R}, \; \; \Sigma_{t} \defeq \big\{ p \in M, \, \hat{t}(p) = t \big\} $$.

Since $\hat{t}$ is regular, the hypersurfaces $\Sigma_{t}$ are non-intersecting: 

\begin{equation*}

    \Sigma_{t} \cap \Sigma_{t'} = \emptyset  \mbox{  for  } t \neq t'

\end{equation*}

Moreover we have that \[ M = \bigcup\limits_{t \in \mathbb{R}} \Sigma_{t}

\]

\subsection{The Normal Evolution vector}

Let's next discuss another timelike vector which is normal to $\Sigma_{t}$ such that: 

\begin{align*}

   \boxed{\vec{m} \defeq \alpha\vec{n}}

\end{align*}

Where since $\vec{n} \cdot \vec{n} &= -1$, we coincidentally get $\vec{m} \cdot \vec{m} = -\alpha^{2}$. 

\newline We also find for the gradient 1-form of the scalar field $t$, and the normal vector $\vec{m}$:

\begin{align}

    \langle \nabla t, \vec{m} \rangle &= \langle \nabla t, \alpha \vec{n} \rangle \notag  \\

    &= \alpha \langle \nabla t, \vec{n} \rangle \notag \\

    &= \alpha \langle \nabla t, - \alpha \overrightarrow{\nabla t} \rangle \notag \\

    &= \alpha^{2}\underbrace{(-\langle \nabla t, \overrightarrow{\nabla t} \rangle)}_{\alpha^{-2}} = 1 

\end{align}

Where we have used $\vec{n} \defeq - \alpha \overrightarrow{\nabla t}$ [cf. Eq.(6.4)] and the definition for $\alpha$.

\newline 

A geometrical consequence of this property is that given a hypersurface $\Sigma_{t}$ and a neighbouring hypersurface $\Sigma_{t+\delta t}$ where $p \in \Sigma_{t}$, $q \in \Sigma_{t+\delta t}$ and $q=p+\delta t \vec{m}$, meaning some point $p$ is displaced by the infinitesimal vector $\delta t \vec{m}$ to point $q$ on $\Sigma_{t+\delta t}$. \newline

Now using the definition of the gradient 1-form $\nabla t$, the value of the scalar field $t$ at point $q$ is:

\begin{align*}

    t(q) &= t(p+\delta t \vec{m}) \\

    &= t(p) + \langle \nabla t, \delta t \vec{m} \rangle \\

    &= t(p) + \delta t \underbrace{\langle \nabla t, \vec{m} \rangle}_{=1} 

    = t(p) + \delta t

\end{align*}

verifying that $q \in \Sigma_{t+\delta t}$. One can say that the hypersurfaces $\Sigma_{t}$ are Lie dragged by the vector $\vec{m}$, meaning that the points on $\Sigma_{t}$ are carried over by the vector $\delta t \vec{m}$ to the neighbouring hypersurface $\Sigma_{t+\delta t}$. We call vector $\vec{m}$ the \textit{normal evolution vector}. An immediate consequence of the Lie dragging property of $\vec{m}$ is that the Lie derivative along $\vec{m}$ of any other vector tangent to $\Sigma_{t}$ is also a vector tangent to $\Sigma_{t}$:

$$\forall \vec{v} \in T(\Sigma_{t}), \, \mathcal{L}_{m} \vec{v} \in T(\Sigma_{t})$$

\subsection{The Lapse Function and Shift vector}

Now you must be thinking how the coordinates between two slices could be related to one another. The answer to your question is in the interpretation of the \textit{lapse} function and the \textit{shift} vector.

\newline

The lapse measures the proper time whilst the shift measures changes in the spatial coordinates. The lapse and the shift describe our choice of coordinates on the hypersurfaces $(\alpha,\beta x,\beta y,\beta z)$.

\begin{figure}[htp]

\centering

\includegraphics[width=10cm]

{"lapse".png}

\begin{center}\textbf{Figure 2:} $n^{\mu}$ is the timelike unit vector normal to the spacelike hypersurface and $\delta t$ is the time step difference between two slices. Adapted from: MatheOverflow

\end{center}

\label{fig:fig}

\end{figure}

One would like to transform a system of tensorial equations into a system of partial differential one for solving purposes. Therefore we introduce coordinates on the spacetime manifold $M$. This is done by introducing a coordinate system $x^{i} = (x^{1},x^{2},x^{3})$ on each hypersurface $\Sigma_{t}$. If this coordinate system changes smoothly (differentiable map) between neighbouring hypersurfaces then we can generalise it to $x^{\mu}=(t,x^{i})=(t,x^{1},x^{2},x^{3})$ as a coordinate system on $M$. Where $x^{i}$ is called the \textbf{spatial coordinates}. Denoting $\vec{\partial}_{\mu}$ the natural basis of $T_{p}(M)$ associated with the coordinates $x^{\mu}$ we get the following set of vectors

\begin{align*}

    &\vec{\partial}_{t} \defeq \frac{\partial}{\partial t} \\ 

  &\vec{\partial}_{i} \defeq \frac{\partial}{\partial x^{i}}, i \in \{1,2,3\}

\end{align*}

Where $\vec{\partial_{i}}$ is tangent to the hypersurface $\Sigma_{t}$ and we call $\vec{\partial_{t}}$ the \textit{time vector}. (Not to be confused with the timelike vector) \newline

The \textbf{shift} evaluates how much the path of a particle at constant spatial coordinates is non-orthogonal to the surface. \newline Let's recall that the dual basis associated with $\vec{\partial}_{\mu}$ is $\vec{d}x^{\mu}$, in particular the 1-form $\vec{d}t=\nabla t$ is dual to the vector $\vec{\partial}_{t}$.

Hence we have that \begin{align}

    \langle \nabla t, \vec{\partial}_{t} \rangle = \langle \vec{d}t, \frac{\partial}{\partial x^{t}} \rangle = \delta^{t}_{t} = 1

\end{align}

From this we can conclude that the time vector $\vec{\partial_{t}}$ follows the same property as our good friend the normal evolution vector $\vec{m}$, since $\langle \nabla t, \vec{m} \rangle = 1$ [cf. Eq.(7.1)]. Therefore we can geometrically conclude that $\vec{\partial}_{t}$ Lie drags the hypersurface $\Sigma_{t}$. The difference between $\vec{\partial}_{t}$ and $\vec{m}$ is called the \textit{shift}, denoted by ($\vec{\beta})$:

\begin{align}

   \boxed{\vec{\partial_{t}} \defeq \vec{m} + \vec{\beta}}

\end{align}

In figure (\textbf{2}) we have for $\vec{m}=\alpha \vec{n}$ and $\vec{\partial}_{t}$ is the natural basis for $t^{\mu}$ where $\mu = 0$.

Coincidentally from Eq.(7.3) we get:

\begin{align*}

    \langle \nabla t, \vec{\beta} \rangle = \langle \nabla t, \vec{\partial}_{t} \rangle - \langle \nabla t, \vec{m} \rangle = 1 - 1 = 0

\end{align*}

Where we have made use of Eqs.(7.1) and (7.2). This result shows us that $\vec{\beta}$ is tangent to our hypersurface $\Sigma_{t}$. It is important to take note that the two vectors $\vec{\partial}_{t}$ and $\vec{m}$ differ from one another and only coincide if the curves of our constant $x^{i}$ are orthogonal to $\Sigma_{t}$ per constant $t$, resulting in $\beta^{\mu}=0$. Following from Eq.(7.3) it is worth re-writing it as: 

\begin{align}

    \vec{\partial}_{t} \defeq \alpha\vec{n} + \vec{\beta}

\end{align}

Since the unit vector $\vec{n}$ is normal to our hypersurface, $\Sigma_{t}$, and $\beta$ is tangent to $\Sigma_{t}$, we can straight forwardly see that Eq.(7.4) is the 3+1 decomposition of the time vector $\vec{\partial}_{t}$ and so we deduce:

$$\vec{\beta} = \overrightarrow{\gamma}(\vec{\partial}_{t})$$ Where $\overrightarrow{\gamma}$ is the Orthogonal projector. We can directly compute from Eq.(7.3) the following:

\begin{align}

    \vec{\partial_{t}} \cdot \vec{\partial_{t}} = -\alpha^{2} + \vec{\beta} \cdot \vec{\beta} 

\end{align}

Since $\vec{\beta}$ is tangent to $\Sigma_{t}$ we can define $\vec{\beta}$ in terms of its natural and dual basis components:

\begin{align*}

    \vec{\beta} \defeq \beta^{i}\vec{\partial_{i}} \; \; \; \;

 \vec{\beta} \defeq \beta_{i}\vec{dx}^{i}\end{align*}

 

\newline The \textbf{lapse ($\alpha$)}  measures the time step iteration between the slices. As discussed we can conclude that the timelike unit vector $\vec{n}$, which is normal to our slice $\Sigma_{t}$, is also collinear to the metric dual vector $\overrightarrow{\nabla} t$. Therefore resulting in the following:

\begin{equation}

    \boxed{\vec{n} \defeq -\alpha \overrightarrow{\nabla}t}

\end{equation}

Where we have \begin{align*}

    \alpha \defeq (\overrightarrow{\nabla} t \cdot \overrightarrow{\nabla} t)^{-1/2} = (- \langle \nabla t, \overrightarrow{\nabla} t \rangle)^{-\frac{1}{2}}

\end{align*}

The value of $\alpha$ ensures that $\vec{n}$ is a unit vector i.e. $\vec{n} \cdot \vec{n} &= -1$. You could also say that the unit vector $\vec{n}$ and the gradient of $t$ are proportional to each other and the constant proportionality is the lapse function.

The minus sign in Eq.(6.4) is chosen by convention to ensure that  $\vec{n}$ is future-oriented.

To be precise, the lapse function never vanishes for a regular foliation (this can be easily seen by the definition of the lapse function and a regular foliation). So you could say that based on Eq.(7.6), $\alpha$ shows the proportionality between the gradient 1-form $\nabla t$ and the 1-form $\underline{\vec{n}}$ which is associated to the vector $\vec{n}$ by basis duality: 

\begin{equation}

    \boxed{\underline{\vec{n}} = -\alpha \nabla t}

\end{equation}

\subsection{Eulerian Observers}

Let us remind ourselves that $\vec{a} \defeq \nabla_{\vec{n}}\underline\vec{n}$. Furthermore, from Eq.(6.3) we deduced that $\vec{a}$ is orthogonal to $\vec{n}$ and so tangent to $\Sigma_{t}$. Additionally, $\vec{a}$ can be expressed in terms of the spatial gradient($D$) of the lapse function thanks to Eq.(7.7):

\begin{align*}

    a_{\nu} = n^{\mu}\nabla_{\mu}n_{\nu} &= -n^{\mu}\nabla_{\mu}(\alpha\nabla_{\nu}t) \\ 

    &= -n^{\mu}(\nabla_{\mu}\alpha)\underbrace{\nabla_{\nu}t}_{- \frac{1}{\alpha} n_{\nu}} - \alpha n^{\mu}\underbrace{(\nabla_{\mu}\nabla_{\nu}t)}_{= \nabla_{\nu}\nabla_{\mu}t} \\ 

    &= \frac{1}{\alpha}n_{\nu}n^{\mu}(\nabla_{\mu}\alpha) - \alpha n^{\mu}\nabla_{\nu}(-\frac{1}{\alpha}n_{\mu}) \\ 

    &= \frac{1}{\alpha}n_{\nu}n^{\mu}\nabla_{\mu}\alpha - \frac{\alpha}{\alpha^{2}}\nabla_{\nu}\alpha \underbrace{n^{\mu}n_{\mu}}_{= -1} + \underbrace{n^{\mu}\nabla_{\nu}n_{\mu}}_{= 0}  \\ 

    &= \frac{1}{\alpha} \big(\nabla_{\nu}\alpha + n_{\nu}n^{\mu}\nabla_{\mu}{\alpha} \big) \intertext{

   where $\nabla_{\nu}$ can be written as $\tensor{\delta}{^\mu_\nu}\nabla_{\mu}$}

    &= \frac{1}{\alpha} \tensor{\gamma}{^\mu_\nu}\nabla_{\mu}\alpha \\

    &= \boxed{\frac{1}{\alpha}D_{\nu}\alpha = D_{\nu} \ln{\alpha}}

\end{align*}

We recall that $\nabla$ is torsion-free and so can permute the indices $\mu,\nu$ in the second line. Also we've made use of expression (6.9) for the orthogonal projector and the relation between $\nabla$ and $\vec{D}$. Thus the final component free results are

\begin{equation}

    \boxed{\underline{\vec{a}} = \vec{D}\ln{\alpha}} \; \; \textrm{and} \; \;  \boxed{\vec{a} = \overrightarrow{\vec{D}}\ln{\alpha}}

\end{equation}

The above is telling us that the 4-acceleration of the Eulerian observer is nothing more than the spatial gradient of the logarithm of the lapse function.


\subsection{Gradients of $n$ and $m$}

If we substitute Eq.(7.8) for $\underline{\vec{a}}$ in Eq.(6.13) then we'll be able to make an association between the extrinsic curvature, the gradient of $\underline{\vec{n}}$ and the spatial gradient of the lapse function:

\begin{equation}

   \vec{\nabla\underline{n}} = -\vec{K} - \vec{D}\ln{\alpha}\otimes\underline{\vec{n}}

\end{equation}

In components

\begin{equation*}

    \boxed{\nabla_{\nu}n_{\mu} = -\tensor{K}{_\mu_\nu} - D_{\mu}\ln{\alpha} \, n_{\nu}}

\end{equation*}

\smallskip

The covariant derivative of the normal evolution vector $\vec{m}$ is derived in the following manner $$\vec{\nabla \underline{m}} = \vec{\nabla}(\alpha\underline{\vec{n}}) = \alpha \vec{\nabla \underline{n}} + \underline{\vec{n}} \otimes \vec{\nabla}\alpha$$

Hence by coinciding the above and Eq.(7.9) we derive the gradient for the normal evolution vector

\begin{equation}

    \vec{\nabla m} = -\alpha \overrightarrow{\vec{K}} - \overrightarrow{\vec{D}}\alpha \otimes \underline{\vec{n}} + \vec{n} \otimes \vec{\nabla}\alpha

\end{equation}

In components

\begin{equation*}

   \boxed{\nabla_{\nu}m^{\mu} = -\alpha\tensor{K}{^\mu_\nu} - D^{\mu}\alpha n_{\nu} + n^{\mu}\nabla_{\nu}\alpha}

\end{equation*}

\subsection{Evolution of the 3-metric}

The evolution of the hypersurface's 3-metric $\vec{\gamma}$ is given by the Lie derivative of $\vec{\gamma}$ along $\vec{m}$. We use Eqs.(7.10) and (3.8) to derive the following:

\begin{align*}

    \mathcal{L}_{\vec{m}} \tensor{\gamma}{_\beta_\nu} &= m^{\mu}\nabla_{\mu}\tensor{\gamma}{_\beta_\nu}+\tensor{\gamma}{_\mu_\nu}\nabla_{\beta}m^{\mu} + \tensor{\gamma}{_\beta_\mu}\nabla_{\nu}m^{\mu} \\

    &= \alpha n^{\mu} \nabla_{\mu} \big(\tensor{g}{_\beta_\nu} + n_{\beta}n_{\nu} \big) - \tensor{\gamma}{_\mu_\nu} \big( \alpha \tensor{K}{^\mu_\beta} + D^{\mu}\alpha n_{\beta} - n^{\mu} \nabla_{\beta}\alpha \big) -\tensor{\gamma}{_\beta_\mu} \big(\alpha\tensor{K}{^\mu_\nu} + D^{\mu}\alpha n_{\nu} -n^{\mu}\nabla_{\nu}\alpha \big) \\

    &= \alpha \Big(\underbrace{n^{\mu}(\nabla_{\mu}n_{\beta})}_{a_{\beta} = \frac{1}{\alpha} D_{\beta}\alpha}n_{\nu} + n_{\beta}\underbrace{n^{\mu}(\nabla_{\mu}n_{\nu})}_{a_{\nu} = \frac{1}{\alpha}D_{\nu}\alpha} \Big) -\alpha\tensor{K}{_\nu_\beta} - D_{\nu}\alpha n_{\beta} + \underbrace{\tensor{\gamma}{_\mu_\nu}n^{\mu}}_{= 0}\nabla_{\beta}\alpha -\alpha\tensor{K}{_\beta_\nu} - D_{\beta}\alpha n_{\nu} \\

    &= \alpha \big(\frac{1}{\alpha} D_{\beta}\alpha n_{\nu} + \frac{1}{\alpha} n_{\beta} D_{\nu} \alpha \big) - 2\alpha\tensor{K}{_\beta_\nu} - D_{\beta}\alpha n_{\nu} -  n_{\beta} D_{\nu} \alpha \\

    &= -2\alpha\tensor{K}{_\beta_\nu}

\end{align*}

Hence 

\begin{equation}

    \boxed{\mathcal{L}_{\vec{m}}\vec{\gamma} = -2\alpha\vec{K}}

\end{equation}

Now one can easily deduce the Lie derivative along of the 3-metric along the unit normal $\vec{n}$ using the relation $\vec{m} = \alpha \vec{n}$,

\begin{align*}

    \mathcal{L}_{\vec{m}} \tensor{\gamma}{_\beta_\nu} &= \mathcal{L}_{\alpha\vec{n}}\tensor{\gamma}{_\beta_\nu} \\ 

    &= \alpha n^{\mu}\nabla_{\mu}\tensor{\gamma}{_\beta_\nu} + \tensor{\gamma}{_\mu_\nu}\nabla_{\beta}(\alpha n^{\mu}) + \tensor{\gamma}{_\beta_\mu}\nabla_{\nu}(\alpha n^{\nu}) \\

    &= \alpha n^{\mu}\nabla_{\mu}\tensor{\gamma}{_\beta_\nu} + \underbrace{\tensor{\gamma}{_\mu_\nu}n^{\mu}}_{= 0}\nabla_{\beta}\alpha + \alpha \tensor{\gamma}{_\mu_\nu}\nabla_{\beta}n^{\mu}+\underbrace{\tensor{\gamma}{_\beta_\mu}n^{\mu}}_{= 0}\nabla_{\nu}\alpha+\alpha\tensor{\gamma}{_\beta_\mu}\nabla_{\nu}n^{\mu} \\ &= \alpha \mathcal{L}_{\vec{n}}\tensor{\gamma}{_\beta_\nu}

\end{align*}

Thus

\begin{equation}

    \mathcal{L}_{\vec{n}}\vec{\gamma} = \frac{1}{\alpha}\mathcal{L}_{\vec{m}}\vec{\gamma}

\end{equation}

Consequently, Eq.(7.11) leads to 

\begin{equation}

    \boxed{\vec{K} = -\frac{1}{2}\mathcal{L}_{\vec{n}}\vec{\gamma}}

\end{equation}

Yet again we have found another interpretation for our good friend the extrinsic curvature $\vec{K}$. In addition to being the gradient of the 1-form $\underline{\vec{n}}$ to which the projector operator is applied to on $\Sigma_{t}$ [cf. Eq.(6.14)] and being the measure of the difference between $\vec{D}$-derivatives and $\vec{\nabla}$-derivatives for vectors tangent to $\Sigma_{t}$ [cf. Eq.(6.17)], we have that $\vec{K}$ is the minus $0.5$ of the Lie derivative along the unit normal of hypersurface's 3-metric.

\subsection{Evolution of the Orthogonal Projector}

Now it's time to evaluate the Lie derivative of the orthogonal projector($\overrightarrow{\vec{\gamma}}$) onto $\Sigma_{t}$ along the normal evolution vector, $\vec{m}$. As before, we use Equations (7.10) and (3.8)

\begin{align*}

        \mathcal{L}_{\vec{m}} \tensor{\gamma}{^\beta_\nu} &= m^{\mu}\nabla_{\mu}\tensor{\gamma}{^\beta_\nu}-\tensor{\gamma}{^\mu_\nu}\nabla_{\mu}m^{\beta} + \tensor{\gamma}{^\beta_\mu}\nabla_{\nu}m^{\mu} \\

    &= \alpha n^{\mu} \nabla_{\mu} \big(\tensor{\delta}{^\beta_\nu} + n^{\beta}n_{\nu} \big) + \tensor{\gamma}{^\mu_\nu} \big( \alpha \tensor{K}{^\beta_\mu} + D^{\beta}\alpha n_{\mu} - n^{\beta} \nabla_{\mu}\alpha \big) -\tensor{\gamma}{^\beta_\mu} \big(\alpha\tensor{K}{^\mu_\nu} + D^{\mu}\alpha n_{\nu} -n^{\mu}\nabla_{\nu}\alpha \big) \\

&= \alpha \Big(\underbrace{n^{\mu}\nabla_{\mu}n^{\beta}}_{\alpha^{-1}D^{\beta}\alpha} n_{\nu} + n^{\beta} \underbrace{n^{\mu}\nabla_{\mu}n_{\nu}}_{\alpha^{-1}D_{\nu}\alpha} \Big) + \alpha\tensor{K}{^\beta_\nu} - n^{\beta}D_{\nu}\alpha -\alpha\tensor{K}{^\beta_\nu} - D^{\beta}\alpha n_{\nu} \\ 

&= 0

\end{align*}

Hence

\begin{equation}

    \boxed{\mathcal{L}_{\vec{m}}\overrightarrow{\vec{\gamma}} = 0}

\end{equation}

This is an important consequence because it shows that taking the Lie derivative along $\vec{m}$ of any tensor field ($\vec{T}$) which is tangent to $\Sigma_{t}$ results in a tensor field which is also tangent to $\Sigma_{t}$, i.e, 

\begin{equation}

    \boxed{\vec{T}  \; \mbox{tangent to} \; \;  \Sigma_{t} \Leftrightarrow \mathcal{L}_{\vec{m}}\vec{T} \; \;  \mbox{tangent to} \; \; \Sigma_{t}}

\end{equation}

To show this explicitly we take a distinctive feature of a tensor field tangent to $\Sigma_{t}$:

\begin{equation}

    \overrightarrow{\vec{\gamma}}^{*} \vec{T} = \vec{T}

\end{equation}

In components (if we take $\vec{T}$ to be a tensor field of type (1,1)) the above equation is

\begin{equation*}

    \tensor{\gamma}{^\alpha_\mu}\tensor{\gamma}{^\nu_\beta}\tensor{T}{^\mu_\nu} = \tensor{T}{^\alpha_\beta}

\end{equation*}

Now we take the Lie derivative along $\vec{m}$ of the above relation, keeping in mind the Leibniz rule and Eq.(7.14)

\begin{align*}

    &\mathcal{L}_{\vec{m}}\big(\tensor{\gamma}{^\alpha_\mu}\tensor{\gamma}{^\nu_\beta}\tensor{T}{^\mu_\nu}\big) = \mathcal{L}_{\vec{m}} \tensor{T}{^\alpha_\beta} \\

    &\underbrace{(\mathcal{L}_{\vec{m}}\tensor{\gamma}{^\alpha_\mu})}_{= 0}\tensor{\gamma}{^\nu_\beta}\tensor{T}{^\mu_\nu} + \tensor{\gamma}{^\alpha_\mu}\underbrace{(\mathcal{L}_{\vec{m}}\tensor{\gamma}{^\nu_\beta})}_{= 0}\tensor{T}{^\mu_\nu} + \tensor{\gamma}{^\alpha_\mu}\tensor{\gamma}{^\nu_\beta} (\mathcal{L}_{\vec{m}}\tensor{T}{^\mu_\nu}) = \mathcal{L}_{\vec{m}} \tensor{T}{^\alpha_\beta} \\

    &\overrightarrow{\vec{\gamma}}^{*}\mathcal{L}_{\vec{m}}\vec{T} = \mathcal{L}_{\vec{m}}\vec{T}.

\end{align*}

This exhibits that $\mathcal{L}_{\vec{m}}\vec{T}$ is tangent to $\Sigma_{t}$. 

\subsection{Last Part of the 3+1 Decomposition of the Riemann Tensor}

In section 6.8, we have successfully managed to form the fully projected 4-dimensional Riemann tensor, i.e. $\overrightarrow{\vec{\gamma}}^{*} \; ^{4}\tensor{R}{^\rho_\sigma_\mu_\nu}$, providing us the Gauss equation [Eq.(6.20)], as well as projecting the 4D Riemann tensor three times onto $\Sigma_{t}$ and once along the normal $\vec{n}$, leading to the Codazzi equation [Eq.(6.21)]. These two decompositions are only meaningful for fields which are tangent to $\Sigma_{t}$ where their derivatives are also parallel to $\Sigma_{t}$, making them only valid for a single hypersurface. Therefore in this section we will proceed to form the projection of the spacetime Riemann tensor which makes sense for a foliation. We do that by projecting $^{4}\tensor{R}{^\mu_\rho_\nu_\sigma}$ twice onto $\Sigma_{t}$ and twice along $\vec{n}$. We will soon see that this will involve a derivative of $K$  which is in a direction \textit{normal} to the hypersurface.

\smallskip

\newline We will start with the Ricci identity applied to the extension of the vector $\vec{n}$. However instead of the full projection, let us project it twice onto $\Sigma_{t}$ and once along $\vec{n}$:

\begin{equation}

    \tensor{\gamma}{_\alpha_\mu}n^{\sigma}\tensor{\gamma}{^\nu_\beta}\, ^{4}\tensor{R}{^\mu_\rho_\nu_\sigma}n^{\rho} = \tensor{\gamma}{_\alpha_\mu}n^{\sigma}\tensor{\gamma}{^\nu_\beta}\big(\nabla_{\nu}\nabla_{\sigma}n^{\mu} - \nabla_{\sigma}\nabla_{\nu}n^{\mu} \big)

\end{equation}

Substituting Eq.(7.9) for $\vec{\nabla n}$:

\begin{align}

   \tensor{\gamma}{_\alpha_\mu}n^{\rho}\tensor{\gamma}{^\nu_\beta}n^{\sigma} \, ^{4}\tensor{R}{^\mu_\rho_\nu_\sigma} &= \tensor{\gamma}{_\alpha_\mu}n^{\sigma}\tensor{\gamma}{^\nu_\beta} \big[-\nabla_{\nu}\big(\tensor{K}{^\mu_\sigma}+D^{\mu}\ln{\alpha}n_{\sigma}\big)+\nabla_{\sigma}\big(\tensor{K}{^\mu_\nu}+\ln{\alpha}n_{\nu} \big)\big] \notag \\

   &= \tensor{\gamma}{_\alpha_\mu}n^{\sigma}\tensor{\gamma}{^\nu_\beta}  \big[ -\nabla_{\nu}\tensor{K}{^\mu_\sigma}-\nabla_{\nu}n_{\sigma}D^{\mu}\ln{\alpha}-n_{\sigma}\nabla_{\nu}D^{\mu}\ln{\alpha} \notag \\

   &\;\;\;\;\;\!\!\!\;\;\;\;\;\;\;\;\;\;\;\;\;\;\;\;\;+\nabla_{\sigma}\tensor{K}{^\mu_\nu}+\nabla_{\sigma}n_{\nu}D^{\mu}\ln{\alpha}+n_{\nu}\nabla_{\sigma}D^{\mu}\ln{\alpha} \big] \\

   &= \tensor{\gamma}{_\alpha_\mu}\tensor{\gamma}{^\nu_\beta}\big[-(\nabla_{\nu}\underbrace{n^{\sigma}\tensor{K}{^\mu_\sigma}}_{0}-\tensor{K}{^\mu_\sigma}\nabla_{\nu}n^{\sigma}) - \underbrace{n^{\sigma}\nabla_{\nu}n_{\sigma}}_{0}D^{\mu}\ln{\alpha} + \nabla_{\nu}D^{\mu}\ln{\alpha} + n^{\sigma}\nabla_{\sigma}\tensor{K}{^\mu_\nu} \notag \\ &\;\;\;\;\;\;\;\;\;\;\;\;\;\;\;\;\;\;\; n^{\sigma}\nabla_{\sigma}n_{\nu}D^{\mu}\ln{\alpha} + n_{\nu}\nabla_{\sigma}n^{\sigma}D^{\mu}\ln{\alpha} \big] \notag \\

   &= \tensor{\gamma}{_\alpha_\mu}\tensor{\gamma}{^\nu_\beta}\big[\tensor{K}{^\mu_\sigma}\nabla_{\nu}n^{\sigma}+\nabla_{\nu}D^{\mu}\ln{\alpha}+n^{\sigma}\nabla_{\sigma}\tensor{K}{^\mu_\nu}+D_{\nu}\ln{\alpha}D^{\mu}\ln{\alpha} \big] \notag \\ 

   &= -\tensor{K}{_\alpha_\sigma}\tensor{K}{^\sigma_\beta} + D_{\beta}D_{\alpha}\ln{\alpha}+ \tensor{\gamma}{^\mu_\alpha}\tensor{\gamma}{^\nu_\beta}n^{\sigma}\nabla_{\sigma}\tensor{K}{_\mu_\nu}+D_{\beta}\ln{\alpha}D_{\alpha}\ln{\alpha} \notag \\

   &= -\tensor{K}{_\alpha_\sigma}\tensor{K}{^\sigma_\beta}+\frac{1}{\alpha}D_{\beta}D_{\alpha}\alpha+\tensor{\gamma}{^\mu_\alpha}\tensor{\gamma}{^\nu_\beta}n^{\sigma}\nabla_{\sigma}\tensor{K}{_\mu_\nu}

\end{align}

Where we have used $n^{\sigma}\nabla_{\sigma}n_{\nu} = D_{\nu}\ln{\alpha}$ and $\tensor{\gamma}{^\nu_\beta}n_{\nu} = 0$ to get the fourth equality. \hl{(REVISIT THE ABOVE DERIVATION- what happened to the last term?)}

Next we want to show that $\tensor{\gamma}{^\mu_\alpha}\tensor{\gamma}{^\nu_\beta}n^{\sigma}\nabla_{\sigma}\tensor{K}{_\mu_\nu}$ is related to $\mathcal{L}_{\vec{m}}\vec{K}$ by using the Lie derivative expansion:

\begin{equation*}

    \mathcal{L}_{\vec{m}}\tensor{K}{_\alpha_\beta} = m^{\mu}\nabla_{\mu}\tensor{K}{_\alpha_\beta} + \tensor{K}{_\mu_\beta}\nabla_{\alpha}m^{\mu}+\tensor{K}{_\alpha_\mu}\nabla_{\beta}m^{\mu}

\end{equation*}

Now substituting Eq.(7.10) for our $\vec{\nabla m}$ and the fact that $\vec{m} \defeq \alpha \vec{n}$ one gets:

\begin{equation*}

    \mathcal{L}_{\vec{m}}\tensor{K}{_\alpha_\beta} = \alpha n^{\mu}\nabla_{\mu}\tensor{K}{_\alpha_\beta} -2\alpha\tensor{K}{_\alpha_\mu}\tensor{K}{^\mu_\beta}-\tensor{K}{_\alpha_\mu}D^{\mu}\alpha n_{\beta} - \tensor{K}{_\beta_\mu}D^{\mu}\alpha n_{\alpha} 

\end{equation*}

Now we would like to project this equation onto $\Sigma_{t}$, meaning that we need to apply the operator $\overrightarrow{\vec{\gamma}}^{*}$ to both sides. Based on property (7.15), we may use the fact that $\overrightarrow{\vec{\gamma}}^{*} \mathcal{L}_{\vec{m}}\vec{K} = \mathcal{L}_{\vec{m}}\vec{K}$:

\begin{align}

    \mathcal{L}_{\vec{m}}\tensor{K}{_\alpha_\beta} &= \tensor{\gamma}{^\mu_\alpha}\tensor{\gamma}{^\nu_\beta}\Big(\alpha n^{\sigma}\nabla_{\sigma}\tensor{K}{_\mu_\nu} -2\alpha \tensor{K}{_\mu_\sigma}\tensor{K}{^\sigma_\nu} - \tensor{K}{_\mu_\sigma}D^{\sigma}\alpha n_{\nu}-\tensor{K}{_\mu_\sigma}D^{\sigma}\alpha n_{\mu} \Big) \notag \\

    &= \alpha \tensor{\gamma}{^\mu_\alpha}\tensor{\gamma}{^\nu_\beta}n^{\sigma}\nabla_{\sigma}\tensor{K}{_\mu_\nu} - 2\alpha\tensor{\gamma}{^\mu_\alpha}\tensor{\gamma}{^\nu_\beta}\tensor{K}{_\mu_\sigma}\tensor{K}{^\sigma_\nu} - \tensor{\gamma}{^\mu_\alpha}\underbrace{\tensor{\gamma}{^\nu_\beta}n_{\nu}}_{0}\tensor{K}{_\mu_\sigma}D^{\sigma}\alpha -\underbrace{\tensor{\gamma}{^\mu_\alpha}n_{\mu}}_{0}\tensor{\gamma}{^\nu_\beta}\tensor{K}{_\mu_\sigma}D^{\sigma}\alpha \notag \\

    &= \alpha \tensor{\gamma}{^\mu_\alpha}\tensor{\gamma}{^\nu_\beta}n^{\sigma}\nabla_{\sigma}\tensor{K}{_\mu_\nu} - 2\alpha\tensor{\gamma}{^\mu_\alpha}\tensor{\gamma}{^\nu_\beta}\tensor{K}{_\mu_\sigma}\tensor{K}{^\sigma_\nu}

\end{align}

Re-writing the above result as: $$\tensor{\gamma}{^\mu_\alpha}\tensor{\gamma}{^\nu_\beta}n^{\sigma}\nabla_{\sigma}\tensor{K}{_\mu_\nu} = \frac{1}{\alpha}\mathcal{L}_{\vec{m}}\tensor{K}{_\alpha_\beta} + 2\tensor{K}{_\alpha_\sigma}\tensor{K}{^\sigma_\beta}$$

and plugging it in Eq.(7.19) we attain the following

\begin{equation}

    \boxed{\tensor{\gamma}{_\alpha_\mu}n^{\rho}\tensor{\gamma}{^\nu_\beta}n^{\sigma} \, ^{4}\tensor{R}{^\mu_\rho_\nu_\sigma} = \frac{1}{\alpha}\mathcal{L}_{\vec{m}}\tensor{K}{_\alpha_\beta}+\frac{1}{\alpha}D_{\alpha}D_{\beta}\alpha+\tensor{K}{_\alpha_\sigma}\tensor{K}{^\sigma_\beta}}

\end{equation}

Note that $\vec{D}$ is torsion-free and hence $D_{\beta}D_{\alpha}\alpha = D_{\alpha}D_{\beta}\alpha$

Equation (7.21) is the last equation we needed to complete the 3+1 decomposition of the spacetime Riemann tensor. It is sometimes called the \textbf{\textit{Ricci equation}} (not to be confused with the Ricci identity.)

Thanks to the partial antisymmetry of the spacetime Riemann tensor, if it is projected three times along $\vec{n}$ it vanishes hence one can project $^{4}\tensor{R}{^\rho_\mu_\nu_\sigma}$ at most twice along $\vec{n}$ to get non-vanishing results. 

\newline In case you haven't already noticed, the left-hand side of the Ricci equation (7.21) is a term which shows up in the contracted Gauss equation (6.21). Combining the two (Subtracting Eq.(6.21) from Eq.(7.21) and solving for simulateous equations) gives rise to a formula which no longer contains the spacetime Riemann tensor, but only the Ricci tensor:

\begin{equation}

    \boxed{\tensor{\gamma}{^\mu_\alpha}\tensor{\gamma}{^\mu_\beta}^{4}\tensor{R}{_\mu_\nu} = -\frac{1}{\alpha}\mathcal{L}_{\vec{m}}\tensor{K}{_\alpha_\beta} - \frac{1}{\alpha}D_{\alpha}D_{\beta}\alpha + \tensor{R}{_\alpha_\beta}+ K\tensor{K}{_\alpha_\beta}-2\tensor{K}{_\alpha_\sigma}\tensor{K}{^\sigma_\beta}}

\end{equation}

In index-free notation the above is:

\begin{equation*}

    \overrightarrow{\vec{\gamma}}^{*} \, ^{4} \vec{R} = -\frac{1}{\alpha}\mathcal{L}_{\vec{m}}\vec{K} - \frac{1}{\alpha} \vec{D}\vec{D}\alpha +\vec{R} + K\vec{K} - 2\vec{K} \cdot \overrightarrow{\vec{K}}

\end{equation*}

Now we'll take the trace of Eq.(7.22) with respect to the metric $\vec{\gamma}$, i.e. by contracting it with $\tensor{\gamma}{^\alpha^\beta}$. On the left-hand side we can limit the range of the indices to spatial ones, i.e. $i,j =\{1,2,3\}$, due to the fact that all the involved tensors are indeed spatial ones. 

\begin{align}

    &\tensor{\gamma}{^\alpha^\beta}\tensor{\gamma}{^\mu_\alpha}\tensor{\gamma}{^\mu_\beta}^{4}\tensor{R}{_\mu_\nu} = \tensor{\gamma}{^i^j}\big(-\frac{1}{\alpha}\mathcal{L}_{\vec{m}}\tensor{K}{_i_j} - \frac{1}{\alpha}D_{i}D_{j}\alpha + \tensor{R}{_i_j}+ K\tensor{K}{_i_j}-2\tensor{K}{_i_l}\tensor{K}{^l_j}\big)\notag \\

    & \;\;\;\;\;\; \Rightarrow  \tensor{\gamma}{^\mu^\nu}\,^{4}\tensor{R}{_\mu_\nu} = -\frac{1}{\alpha}\tensor{\gamma}{^i^j}\mathcal{L}_{\vec{m}}\tensor{K}{_i_j} - \frac{1}{\alpha} D_{i}D^{i}\alpha + R + K^{2} -2\tensor{K}{_i_j}\tensor{K}{^i^j}

\end{align}

Now we separately evaluate 

$\tensor{\gamma}{^\mu^\nu}\,^{4}\tensor{R}{_\mu_\nu}$ to get: 

$$\tensor{\gamma}{^\mu^\nu}\,^{4}\tensor{R}{_\mu_\nu} = (\tensor{g}{^\mu^\nu}+n^{\mu}n^{\nu})^{4}\tensor{R}{_\mu_\nu} =\,^{4}R +\,^{4}\tensor{R}{_\mu_\nu}n^{\mu}n^{\nu}$$

and also $$-\tensor{\gamma}{^i^j}(\mathcal{L}_{\vec{m}}\tensor{K}{_i_j}) = -\mathcal{L}_{\vec{m}}(\underbrace{\tensor{\gamma}{^i^j}\tensor{K}{_i_j}}_{K})+\tensor{K}{_i_j}\mathcal{L}_{\vec{m}}\tensor{\gamma}{^i^j}$$

Next we proceed to evaluating $\mathcal{L}_{\vec{m}}\tensor{\gamma}{^i^j}$ using the very definition of the inverse 3-metric:

\begin{align}

    & \; \; \; \; \; \; \; \; \; \;  \tensor{\gamma}{_i_l}\tensor{\gamma}{^l^j} = \tensor{\delta}{^j_i} \notag \\

    &\Rightarrow \mathcal{L}_{\vec{m}}(\tensor{\gamma}{_i_l})\tensor{\gamma}{^l^j} +\tensor{\gamma}{_i_l} \mathcal{L}_{\vec{m}}\tensor{\gamma}{^l^j} = \mathcal{L}_{\vec{m}}(\tensor{\delta}{^j_i}) \notag \\

    &\Rightarrow \tensor{\gamma}{^i^k}\tensor{\gamma}{^l^j}\mathcal{L}_{\vec{m}}\tensor{\gamma}{_k_l} + \underbrace{\tensor{\gamma}{^i^k}\tensor{\gamma}{_k_l}}_{\tensor{\delta}{^i_l}}\mathcal{L}_{\vec{m}}\tensor{\gamma}{^k^j} = 0 \notag \\

    &\Rightarrow \mathcal{L}_{\vec{m}}\tensor{\gamma}{^i^j} = -\tensor{\gamma}{^i^l}\tensor{\gamma}{^j^k}\mathcal{L}_{\vec{m}}\tensor{\gamma}{_l_k} \notag \\

    &\Rightarrow \mathcal{L}_{\vec{m}} \tensor{\gamma}{^i^j} = 2\alpha\tensor{\gamma}{^i^l}\tensor{\gamma}{^j^k}\tensor{K}{_l_k} \notag \\

    &\Rightarrow \boxed{\mathcal{L}_{\vec{m}}\tensor{\gamma}{^i^j} = 2\alpha\tensor{K}{^i^j}}

\end{align}

Where we have used Eq.(7.11). \hl{REVISIT ABOVE}

We plug in Eq.(7.24) back into the derivation for $-\tensor{\gamma}{^i^j}(\mathcal{L}_{\vec{m}}\tensor{K}{_i_j})$ to finally get

\begin{equation}

    -\tensor{\gamma}{^i^j}(\mathcal{L}_{\vec{m}}\tensor{K}{_i_j}) = -\mathcal{L}_{\vec{m}}K + 2\alpha\tensor{K}{_i_j}\tensor{K}{^i^j}.

\end{equation}

Therefore Eq.(7.23) becomes

\begin{equation}

   ^{4}R +\,^{4}\tensor{R}{_\mu_\nu}n^{\mu}n^{\nu} = R + K^{2} - \frac{1}{\alpha}\mathcal{L}_{\vec{m}}K -\frac{1}{\alpha}D_{i}D^{i}\alpha

\end{equation}

Lastly we combine the above with the scalar Gauss relation Eq.(6.22) in order to get rid of the Ricci tensor term $^{4}\tensor{R}{_\mu_\nu}n^{\mu}n^{\nu}$ and acquire an equation which only involves the Ricci scalar $^{4}R$:

\begin{equation*}

\begin{cases}

2 \big(^{4}R +\,^{4}\tensor{R}{_\mu_\nu}n^{\mu}n^{\nu} = R + K^{2} - \frac{1}{\alpha}\mathcal{L}_{\vec{m}}K -\frac{1}{\alpha}D_{i}D^{i}\alpha \big) \\

- \big(^{4}R + 2 ^{4}\tensor{R}{_\mu_\nu}n^\mu n^\nu = R + K^{2} - \tensor{K}{_i_j}\tensor{K}{^i^j} \big)

\end{cases}

\end{equation*}

Consequently, we get

\begin{equation}

   \boxed{^{4}R = R + K^{2} + \tensor{K}{_i_j}\tensor{K}{^i^j} - \frac{2}{\alpha}\mathcal{L}_{\vec{m}}K-\frac{2}{\alpha}D_{i}D^{i}\alpha}

\end{equation}

\subsection{3+1 Decomposition of the Stress-Energy Tensor}

From the very definition of the stress-energy tensor we will derive the following: \begin{itemize}

    \item The matter energy density

    \item The matter momentum density

    \item The matter stress tensor

\end{itemize} The \textbf{matter energy density} is obtained by the \textit{time-time projection} of $\tensor{T}{_\mu_\nu}$:

\begin{align}

   E = \tensor{T}{_\mu_\nu}n^{\mu}n^{\nu} 

\end{align}

where $E$ is a scalar and we can re-write the above as:

$$\boxed{E \defeq \vec{T}(\vec{n},\vec{n})}$$

\newline This follows from the fact that the 4-velocity of the Eulerian observer is the unit vector $\vec{n}$ normal to the hypersurface $\Sigma_{t}$. As for the \textbf{matter momentum density}, it is the \textit{time-spatial} projection of $\tensor{T}{_\mu_\nu}$:

\begin{align}

    P_{i} = -\tensor{T}{_\mu_\nu}n^{\mu}\tensor{\gamma}{^\nu_i}

\end{align}

When measured by the Eulerian observer, it takes the linear form:

$$\boxed{\vec{P} \defeq -T(\vec{n},\overrightarrow{\vec{\gamma}})}$$

i.e. $\forall \vec{v} \in T_{p}(M), \; \; \; \langle \vec{p} , \vec{n} \rangle = -\vec{T}(\overrightarrow{\vec{\gamma}}(\vec{v}),\vec{n})$.

Finally we derive the \textbf{matter stress tensor} which can also directly be obtained by taking the \textit{spatial-spatial} projection of $\tensor{T}{_\mu_\nu}$:

\begin{equation}

    \tensor{S}{_i_j} = \tensor{T}{_\mu_\nu}\tensor{\gamma}{^\mu_i}\tensor{\gamma}{^\nu_j}

\end{equation}

When measured by the Eulerian observer it takes the bilinear form: $$\boxed{\vec{S} \defeq \overrightarrow{\gamma}^{*}\vec{T}}$$

Both $P_{i}$ and $\tensor{S}{_i_j}$ are tangent to our hypersurface $\Sigma_{t}$. 

Now recall $\overrightarrow{\vec{\gamma}}$ written out in its full components [cf Eq.(6.10)]: $$\gamma^{\alpha} \, _{\beta} &= \delta^{\alpha} \,  _{\beta} + n^{\alpha} \, n_{\beta} \Rightarrow \boxed{ \tensor{\delta}{^\alpha_\beta} &= \tensor{\gamma}{^\alpha_\beta} - n^{\alpha}n_{\beta}}$$ and we substitute both equations back into Eq.(7.8) to derive the following:

\begin{align}

    \tensor{S}{_\alpha_\beta} &= \tensor{T}{_\mu_\nu}(\tensor{\delta}{^\mu_\alpha} + n^{\mu}n_{\alpha}) (\tensor{\delta}{^\nu_\beta}+n^{\nu}n_{\beta}) \notag \\  &= \tensor{T}{_\mu_\nu}\tensor{\delta}{^\mu_\alpha}\tensor{\delta}{^\nu_\beta} + \tensor{T}{_\mu_\nu}n^{\mu}n_{\alpha}n^{\nu}n_{\beta} +  \tensor{T}{_\mu_\nu}\tensor{\delta}{^\mu_\alpha}n^{\nu}n_{\beta}+\tensor{T}{_\mu_\nu}\tensor{\delta}{^\nu_\beta}n^{\mu}n_{\alpha} \notag \\ &= \tensor{T}{_\alpha_\beta} + \tensor{T}{_\mu_\nu}n^{\mu}n_{\alpha}n^{\nu}n_{\beta} + \tensor{T}{_\mu_\nu}(\tensor{\gamma}{^\mu_\alpha}-n^{\mu}n_{\alpha})n^{\nu}n_{\beta} + \tensor{T}{_\mu_\nu}(\tensor{\gamma}{^\nu_\beta}-n^{\nu}n_{\beta})n^{\mu}n_{\alpha} \notag \\ &= \tensor{T}{_\alpha_\beta} + \tensor{T}{_\mu_\nu}n^{\mu}n_{\alpha}n^{\nu}n_{\beta} + (\tensor{T}{_\mu_\nu}\tensor{\gamma}{^\mu_\alpha}n^{\nu})n_{\beta} - \tensor{T}{_\mu_\nu}n^{\mu}n_{\alpha}n^{\nu}n_{\beta} + (\tensor{T}{_\mu_\nu}\tensor{\gamma}{^\nu_\beta}n^{\mu})n_{\alpha} - (\tensor{T}{_\mu_\nu}n^{\mu}n^{\nu})n_{\beta}n_{\alpha} \notag \\

    &\Rightarrow \tensor{S}{_\alpha_\beta} = \tensor{T}{_\alpha_\beta} + \underbrace{(\tensor{T}{_\mu_\nu}\tensor{\gamma}{^\mu_\alpha}n^{\nu})}_{=-P_{\alpha}}n_{\beta} + \underbrace{(\tensor{T}{_\mu_\nu}\tensor{\gamma}{^\nu_\beta}n^{\mu})}_{=-P_{\beta}}n_{\alpha} - \underbrace{(\tensor{T}{_\mu_\nu}n^{\mu}n^{\nu})}_{= E}n_{\beta}n_{\alpha} \notag \\ &\Rightarrow \tensor{S}{_\alpha_\beta} = \tensor{T}{_\alpha_\beta} - P_{\alpha}n_{\beta} - P_{\beta}n_{\alpha}- E \, n_{\beta}n_{\alpha}  \notag \\ &\Rightarrow \tensor{T}{_\alpha_\beta} = \tensor{S}{_\alpha_\beta} + P_{\alpha}n_{\beta} + P_{\beta}n_{\alpha} +

    E n_{\alpha}n_{\beta}

\end{align}

We take the trace of Eq.(7.9) by contracting it with $\tensor{g}{^\alpha^\beta}$ and get:

\begin{align*}

  \vec{T} = \vec{S} + \vec{n} \bigotimes \vec{p} +\vec{p} \bigotimes \vec{n} + E \vec{n} \bigotimes \vec{n}

\end{align*}

You could say that the above constitutes the 3+1 decomposition of $\vec{T}$. Then we re-write it as:

\begin{align*}

    \vec{T} = \vec{S} + 2 \underbrace{\langle \vec{n}, \vec{p} \rangle}_{= 0} + E \underbrace{\langle \vec{n}, \vec{n} \rangle}_{= -1} 

\end{align*}

Where we finally achieve the following:

\begin{equation}

    \boxed{T = S - E}

\end{equation}

\subsection{Lie derivatives Along $m$ as Partial Derivatives}

We remind ourselves of Eq.(7.3) and re-write it as $\vec{m} = \vec{\partial}_{t} - \vec{\beta}$. Now let $\vec{T}$ be any tensor field tangent to $\Sigma_{t}$ then we attain the following  

\begin{equation}

   \mathcal{L}_{\vec{m}} \; \vec{T} = \mathcal{L}_{\vec{\partial}t} \; \vec{T} - \mathcal{L}_{\vec{\beta}} \; \vec{T}

\end{equation}

Since $\vec{\beta}$ and $\vec{T}$ are both tangent to $\Sigma_{t}$ then one can conclude that $\mathcal{L}_{\vec{\beta}} \; \vec{T}$

is also tangent to our hypersurface. Furthermore, we know that $\mathcal{L}_{\vec{m}} \; \vec{T}$ is also tangent to $\Sigma_{t}$ due to statement (7.15). Due to these one can also easily conclude that $\mathcal{L}_{\vec{\partial}t} \; \vec{T}$ is also tangent to $\Simga_{t}$. 

Additionally, if one adapts tensor components with respect to the coordinate system on the foliation, $x^{\mu} = (t,x^{i})$ then the Lie derivate taken along $\vec{\partial}_{t}$ on $\Sigma_{t}$ is simply the partial derivative with respect to $t$ [cf.Eq.(3.8)]:

\begin{equation*}

    \mathcal{L}_{\vec{\partial}_{t}}\tensor{T}{^i^\dots_j_\dots} = \frac{\partial}{\partial t}\tensor{T}{^i^\dots_j_\dots}

\end{equation*}

Therefore with respect to foliated coordinates Eq.(7.21) can be written as

\begin{equation}

    \boxed{\mathcal{L}_{\vec{m}}\tensor{T}{^i^\dots_j_\dots} = \Big(\frac{\partial}{\partial t} - \mathcal{L}_{\vec{\beta}} \Big) \tensor{T}{^i^\dots_j_\dots}}

\end{equation}

Later on we will put this formula into good use when finding the Lie derivative of the extrinsic curvature along $\vec{m}$ which is the term $\mathcal{L}_{\vec{m}}\vec{K}$ in the 3+1 Einstein equation:

\begin{equation}

    \mathcal{L}_{\vec{m}}\tensor{K}{_i_j} = \Big(\frac{\partial}{\partial t} - \mathcal{L}_{\beta} \Big) \tensor{K}{_i_j}

\end{equation}

According to formula (3.9) we express $\mathcal{L}_{\beta}\tensor{K}{_i_j}$ in terms of partial derivatives:

\begin{equation}

    \mathcal{L}_{\beta}\tensor{K}{_i_j} = \beta^{\alpha}\frac{\partial \tensor{K}{_i_j}

    }{\partial x^{\alpha}} + \tensor{K}{_\alpha_j}\frac{\partial \beta^{\alpha}}{\partial x^{i}} + \tensor{K}{_i_\alpha}\frac{\partial \beta^{\alpha}}{\partial x^{j}}

\end{equation}

Applying Equation (7.23) and the relation between $\mathcal{L}_{\vec{m}}\vec{\gamma}$ and $\vec{K}$ in (7.11) we get:

\begin{equation}

    \Big(\frac{\partial}{\partial t} - \mathcal{L}_{\vec{\beta}} \Big)\tensor{\gamma}{_i_j} = -2\alpha\tensor{K}{_i_j}

\end{equation}

Again administering formula (3.9) with connection $\vec{D}$ one gets:

\begin{align}

    \mathcal{L}_{\beta}\tensor{\gamma}{_i_j} &= \beta^{\alpha}\underbrace{D_{\alpha}\tensor{\gamma}{_i_j}}_{0} + \tensor{\gamma}{_\alpha_j}D_{i}\beta^{\alpha}+\tensor{\gamma}{_i_\alpha}D_{j}\beta^{\alpha} \notag \\

    &= D_{i}\beta_{j} + D_{j}\beta_{i}

\end{align}

\section{Solving for a Boson star}

The equations leading to boson-star solutions involve the Einstein equations for the geometry and the Klein-Gordon equation to represent the (complex) scalar field. This coupled system is called the Einstein-Klein-Gordon (EKG) equations.

\subsection{Einstein-Klein-Gordon Wave equation}

The action which derives the EKG evolution equations (Wald 1984) is the following

\begin{equation}

    S = \int (\frac{1}{16 \pi G}R+\mathcal{L}_{M})\sqrt{-g}\, d^4x

\end{equation}

where $R$ denotes the Ricci scalar, and $\sqrt{-g}$ is the determinant of the metric $g_{\mu\nu}$ of the spacetime. The term $\mathcal{L}_{M}$ describes the matter, which in this case lives in the complex scalar field, $\Phi$ and $\overline{\Phi}$ is the complex conjugate of the field.

\begin{equation*}

    \mathcal{L}_{M} = -\frac{1}{2}\Big[\tensor{g}{^\mu^\nu} \nabla_{\mu}\overline{\phi} \nabla_{\nu}\phi+ V\big(|\phi|^{2} \big) \Big]

\end{equation*}

Taking the equation of motion of action (8.1) with respect to the metric, i.e. $\frac{\partial S}{\partial \tensor{g}{_\mu_\nu}}=0$, leads to the Einstein's equations:

\begin{equation}

    \tensor{R}{_\mu_\nu}-\frac{R}{2}\tensor{g}{_\mu_\nu} = 8\pi G \, \tensor{T}{_\mu_\nu}

\end{equation} \begin{equation} 

    \tensor{T}{_\mu_\nu}=\frac{1}{2}\big(\nabla_{\mu}\overline{\phi} \nabla_{\nu}\phi+\nabla_{\mu}\phi \nabla_{\nu}\overline{\phi}\big)-\frac{1}{2}\tensor{g}{_\mu_\nu}\big(\tensor{g}{^\rho^\tau}\nabla_{\rho}\overline{\phi}\nabla_{\tau}\phi\big)\end{equation}

where we have $R_{\mu\nu}$ to be the Ricci tensor and $T_{\mu\nu}$ the stress-energy tensor.

\newline

Furthermore the equation of motion of Eq.(8.1) becomes the free wave equation for a scalar field, namely the Klein-Gordon equation:

\begin{equation}

    \tensor{g}{^\mu^\nu}\nabla_{\mu}\nabla_{\nu}\phi = \frac{dV}{d|\phi|^{2}}\phi

\end{equation}

We can obtain an equivalent equation if we vary Eq.(8.1) with respect to the complex conjugate of our scalar field $\overline{\phi}$. The simplest case leading to boson star solutions is the free field case, where the potential is the following

\begin{equation}

    V\big(|\phi|^{2} \big) = m^{2}|\phi|^{2}

\end{equation}

$m$ is a parameter and is identified with the mass of the field theory. It's also good to know that the potential depends on the magnitude of the scalar field.


\subsection{3+1 Decomposition of Einstein's Field Equations} So far we have introduced Einstein's Field Equations (EFE) in the following form:

\begin{equation}

    \tensor{R}{_\mu_\nu}-\frac{1}{2}R\tensor{g}{_\mu_\nu}=8\pi \tensor{T}{_\mu_\nu}

\end{equation}

Where we have assumed the universe to be non-expanding, i.e. $\Lambda = 0$.

What follows is the trace-reversed form of EFE by contracting Eq.(8.6) with the inverse metric $\tensor{g}{^\mu^\nu}$:

\begin{align}

    &\tensor{g}{^\mu^\nu}(\tensor{R}{_\mu_\nu}-\frac{1}{2}R\tensor{g}{_\mu_\nu} = 8\pi \tensor{T}{_\mu_\nu}) \notag \\

    &\Rightarrow R - \frac{1}{2} R \cdot 4 = 8\pi T \notag \\

    &\Rightarrow R - 2R = 8\pi T \notag \\

    &\Rightarrow R = -8\pi G T

\end{align}

Now we substitute back $R = -8\pi T$ into Eq.(8.1) to get:

\begin{align}

    &\tensor{R}{_\mu_\nu} - \frac{1}{2} (-8 \pi G T) \tensor{g}{_\mu_\nu} = 8\pi G \tensor{T}{_\mu_\nu} \notag \\

    &\Rightarrow \tensor{R}{_\mu_\nu} + 4\pi G T\tensor{g}{_\mu_\nu} = 8\pi G \tensor{T}{_\mu_\nu} \notag \\

    &\Rightarrow \boxed{\tensor{R}{_\mu_\nu} = 8\pi G (\tensor{T}{_\mu_\nu} - \frac{1}{2}T\tensor{g}{_\mu_\nu})}

\end{align}

Where $T$ represents the trace of the stress-energy tensor with respect to $\vec{g}$.\smallskip \newline You can't deny that the space-time description of general relativity looks very elegant and simple. However these equations mask a great deal of complexity and when written out in full form (i.e. expressed in terms of the four generalised coordinates of spacetime) they can contain thousands of terms. Moreover the covaraint form of EFE is not suitable to dictate how an initial structure evolves over time. Therefore it is more intuitive to recast EFE in the form of computable, time-step iteration processes by decomposing the space-time. This is done by creating a stack of 3D spacelike "foliations"(as mentioned in Section (7.2)) where each slice is characterised by a fixed time coordinate. Hence we instead consider a succession of spacetime geometries, where the evolution of a given slice is determined by EFE and the system is evolving by moving with time from one foliation to the next. In order to decompose our 4D, covariant EFE to "space + time" we sum up everything we have encountered so far in the following steps:

\begin{enumerate}

    \item \textbf{Specifying the choice of coordinates:} 

    As mentioned above, the spacetime is foliated by a succession of stacked up 3D hypersurfces which are related to one another by the congruence of time-lines that will determine our Eulerian observers (i.e. observers). This congruence is described by the vector field $t^{\mu} = \alpha n^{\mu} + \beta^{\mu}$ where $\alpha$ is our good old friend the lapse function and $\beta^{\mu}$ is the shift vector.

    \item \textbf{Decomposition of every 4D object into 3+1 components:} Everything that we have learned so far in the past two chapters will now come into play for our main purpose. As we have previously observed, any four-dimensional tensor can be decomposed into 3+1 parts using the spatial projector to obtain spatial components or contracting with $n^{\mu}$ for our time components. \newline The general line-element of the spacetime metric has the form:

    \begin{equation*}

        ds^{2} = -\alpha^{2}dt^{2} + \tensor{\gamma}{_i_j}(dx^{i}+\beta^{i}dt)(dx^{j}+\beta^{j}dt)

    \end{equation*}

    where $d\sigma^{2}=\tensor{\gamma}{_i_j}dx^{i}dx^{j}$ is the spacetime geometry metric.

\end{enumerate}

\subsubsection{Einstein's Evolution vs. Constraint Equations}

In our "3+1" split of EFE, the constraint equations must remain time-independent whilst the evolution equations, which provide the evolution of the initial spacetime, are time-dependent. This means that when you do apply time and space projections on EFE you end up with some evolution (time dependent) and so constraint (time independent) equations.

\subsubsection{Projection of the Einstein Equation}

Now that we know the 3+1 decpmposition of the spacetime Ricci tensor and the stress-energy tensor, we are fully equipped to write down the field equations in terms of the extrinsic curvature and it's trace, $\textrm{trK} = \tensor{K}{_i^i}$. \newline Note that the induced metric $\tensor{\gamma}{_i_j}$ and scalar field $\phi$ are as yet still unknown and that the shift and lapse just describe our choice of coordinates. 

\begin{itemize}

    \item \textit{\textbf{Full projection onto $\Sigma_{t}$}} \smallskip

    \newline We apply the operator $\overrightarrow{\vec{\gamma}}^{*}$ to the trace-reversed Einstein equation [cf. Eq.(8.7)].

    \begin{equation}

        \overrightarrow{\vec{\gamma}}^{*} ^{4}\vec{R} = 8 \pi G \Big(\overrightarrow{\vec{\gamma}}^{*}\vec{T} - \frac{1}{2}T \overrightarrow{\vec{\gamma}}^{*}\vec{g} \Big)

    \end{equation}

    Where the $\overrightarrow{\vec{\gamma}}^{*} ^{4}\vec{R}$ is given by Eq.(7.22) which is the combination of the contracted Gauss equation with the Ricci equation, $\overrightarrow{\vec{\gamma}}^{*}\vec{T}$ is the defintion of $\vec{S}$ [Eq.(7.3)], $T = S-E$ [Eq.(7.32)] and finally $\overrightarrow{\vec{\gamma}}^{*}\vec{g} = \vec{\gamma}$.

    Hence one gets:

    \begin{equation}

        \mathcal{L}_{\vec{m}}\vec{K} = -\vec{D}\vec{D}\alpha + \alpha\Big\{ \vec{R} + K\vec{K} - 2\vec{K} \cdot \overrightarrow{\vec{K}} = 4 \pi G \Big[(S-E)\vec{\gamma}- 2\vec{S} \Big]\Big\}

    \end{equation}

    Each term in the above equation is a tensor field tangent to $\Sigma_{t}$ therefore we may restrict the indices to spatial ones without loss of generality and write Eq.(8.18) as:

    \begin{align}

        \boxed{\big(\partial_{t} - \mathcal{L}_{\beta} \big)\tensor{K}{_i_j} = -D_{i}D_{j}\alpha + \alpha  \big\{ \tensor{R}{_i_j} - 2\tensor{K}{_i^l}\tensor{K}{_j_l} + \textrm{trK} \tensor{K}{_i_j} -8 \pi G \big[ \tensor{S}{_i_j} - \frac{\tensor{\gamma}{_i_j}}{2}(S-E) \big] \big\}}

    \end{align}

    Where we have made use of Eq.(7.35). 

\newline You could say that $\tensor{K}{_i_j}$ is introduced to describe the change of the induced metric along the congruence of observers, i.e. $\tensor{K}{_i_j} \equiv -(\frac{1}{2})\mathcal{L}_{\vec{m}}\tensor{\gamma}{_i_j} = -1/(2\alpha) \big(\partial_{t} -\mathcal{L}_{\beta} \big)\tensor{\gamma}{_i_j}$, therefore we introduce a second evolution equation which we will make use of later on

\begin{equation}

    \boxed{\Big(\frac{\partial}{\partial t} - \mathcal{L}_{\vec{\beta}} \Big)\tensor{\gamma}{_i_j} = -2\alpha\tensor{K}{_i_j}}

\end{equation}

    \item \textit{\textbf{Mixed projection}} \smallskip

\newline We will project the trace-reversed Einstein equation once onto $\Sigma_{t}$ and one along the normal $\vec{n}$:

    \begin{equation}

        ^{4}\vec{R}(\vec{n},\overrightarrow{\vec{\gamma}}) - \frac{1}{2}\; ^{4}R \underbrace{\vec{g}(\vec{n},\overrightarrow{\vec{\gamma}})}_{0} = 8\pi G \vec{T}(\vec{n},\overrightarrow{\vec{\gamma}})

    \end{equation}

    We remind ourselves of the contracted Codazzi equation (6.27) and $\vec{T}(\vec{n},\overrightarrow{\vec{\gamma}}) = -\vec{P}$. We therefore get 

    \begin{equation}

        \boxed{D_{j}\tensor{K}{^j_i}-D_{i}\textrm{trK} = 8 \pi G P_{i}}

    \end{equation}

  This equation is also called the \textit{momentum constraint}.  

    \item \textbf{\textit{Full projection perpendicular to $\Sigma_{t}$}} \smallskip

    \newline Finally we apply the trace-reversed form of Einstein's equation to the couple $(\vec{n},\vec{n})$. Note that Einstein's equation is an identity between bilinear forms.

    \begin{equation}

        ^{4}\vec{R}(\vec{n},\vec{n}) - \frac{1}{2} \; ^{4}R \vec{g}(\vec{n},\vec{n}) = 8 \pi G \vec{T}(\vec{n},\vec{n}) 

    \end{equation}

    Using the scalar Gauss equation (6.22), and Eq.(7.28) for $\vec{T}(\vec{n},\vec{n})$ as well as  $\vec{g}(\vec{n},\vec{n}) = -1$, one gets:

    \begin{equation}

        \boxed{\tensor{R}{_i^i} + (\textrm{trK})^{2} - \tensor{K}{_i^j}\tensor{K}{_j^i} = 16 \pi G E} 

    \end{equation}

    This equation is called the \textit{Hamiltonian constraint.}

\end{itemize}

\subsection{3+1 Writing of Metric Components}

The components $g_{\mu}_{\nu}$ of the bilinear form $\vec{g}$ in terms of its dual basis with respect to the coordinates $x^{\mu}$ can be written as: $$\vec{g} := g_{\mu}_{\nu}\vec{d}x^{\mu}\otimes\vec{d}x^{\nu}$$

Similarly we can introduce the components $\gamma_{\alpha}_{\beta}$ of the 3-metric $\vec{\gamma}$ with respect to the spatial coordinates $x^i$

$$\vec{\gamma} := \gamma_{i}_{j}\vec{d}x^{i}\otimes\vec{d}x^{j}$$.

\newline

The 4-dimensional metric $\vec{g}$ on a real n-dimensional vector space $v$ is by definition a \textit{bilinear form}. Therefore we have that

\begin{align*}

    \vec{g} \colon v \times v &\to \mathbb{R} \\

                   (u,w) &\mapsto  \vec{g}(u,w)

\end{align*}

As $\vec{g}$ is bilinear, it is enough to know its action on basis vectors $\vec{\partial}_{\mu}$ and hence we have that \begin{align*}

    g_{\mu}_{\nu} = \vec{g} (\vec{\partial}_{\mu},\vec{\partial}_{\nu})

\end{align*}

Moreover thanks to Eq.(7.5) we can derive the following:


\begin{align}

    g_{0}_{0} = \vec{g}(\vec{\partial}_{t},\vec{\partial}_{t})=\vec{\partial}_{t} \cdot \vec{\partial}_{t} = -\alpha^{2} + \vec{\beta} \cdot \vec{\beta} = -\alpha^{2} + \beta_{i}\beta^{i}

\end{align}

To derive $g_{0}_{i}$, we remind ourselves of $\vec{\partial_{t}}=\vec{m}+\vec{\beta}$ and so conclude that \begin{align}

    g_{0}_{i} = \vec{g}(\vec{\partial}_{t},\vec{\partial}_{i}) = (\vec{m}+\vec{\beta}) \cdot \vec{\partial}_{i}

\end{align}

We expand Eq.(7.3) using the facts that $\vec{m}$ is normal and $\vec{\partial}_{i}$ is tangent to $\Sigma_{t}$ resulting in $\vec{m} \cdot \vec{\partial}_{i} = 0$ moreover we expand $\vec{\beta}$ in terms of its dual basis vector $\vec{\beta} = \beta_{j} \vec{d}^{j}$. Thus

\begin{align}

    g_{0}_{i} = (\vec{m}+\vec{\beta}) \cdot \vec{\partial}_{i} &= \langle \vec{\beta}, \vec{\partial}_{i} \rangle \notag\\ &= \langle \beta_{j} \vec{d}x^{j}, \vec{\partial}_{i} \rangle \notag\\ &= \beta_{j} \underbrace{\langle \vec{d}x^{j}, \vec{\partial}_{i} \rangle}_{\delta^{j}_{i}} \notag \\ &= \beta_{i}

\end{align}

To deduce $g_{i}_{j}$ we follow the same principle, deriving $\vec{g}(\vec{\partial}_{i},\vec{\partial}_{j})$ with the insight that both $\vec{\partial}_i$ and $\vec{\partial}_j$ are tangent to $\Sigma_{t}$ and using the very definition of the bilinear form of the induced spatial metric $\vec{\gamma}$ on $\Sigma_{t}$ we derive

\begin{align}

   \tensor{g}{_i_j} = \vec{g}(\vec{\partial}_{i},\vec{\partial}_{j})=\vec{\gamma}(\vec{\partial}_{i},\vec{\partial}_{j})=\tensor{\gamma}{_i_j}

\end{align}

Putting together Eqs.(8.16), (8.18) and (8.19) we finally attain the 4D metric component $\tensor{g}{_\mu_\nu}$ in terms of 3+1 quantities:

\begin{equation}

   \tensor{g}{_\mu_\nu} = 

\begin{pmatrix}

\tensor{g}{_0_0} & \tensor{g}{_0_j} \\

\tensor{g}{_i_0} & \tensor{g}{_i_j}

\end{pmatrix}

= 

\begin{pmatrix}

-\alpha^{2}+\beta_{k}\beta^{k} & \beta_{j} \\

\beta_{i} & \gamma_{i}_{j}

\end{pmatrix}

\end{equation}

Resulting in the line element:

\begin{equation}

   \boxed{ds^{2}=\tensor{g}{_\mu_\nu}dx^{\mu}dx^{\nu}=-\alpha^{2}dt^{2}+\tensor{\gamma}{_i_j}(dx^{i}+\beta^{i}dt)(dx^{j}+\beta^{j}dt)}

\end{equation}

The inverse metric given by the matrix inverse of (8.8) is the following:

\begin{equation}

    \tensor{g}{^\mu^\nu} =

    \begin{pmatrix}

    \tensor{g}{^0^0} & \tensor{g}{^0^j} \\

    \tensor{g}{^i^0} & \tensor{g}{^i^j}

    \end{pmatrix}

    =

    \begingroup

    \renewcommand*{\arraystretch}{1.5}

    \begin{pmatrix}

    -\frac{1}{\alpha^{2}} & \frac{\beta^{j}}{\alpha^{2}} \\

    \frac{\beta^{i}}{\alpha^{2}} & \gamma^{ij}-\frac{\beta^{i}\beta^{j}}{\alpha^{2}}

    \end{pmatrix}

    \endgroup

\end{equation}

\subsection{The 3+1 decomposition of the spacetime coupled with a metric and the Klein-Gordon wave equation}

\newline From Eq.(8.4) one can conclude that the Klein-Gordon wave equation is second order in time and space. Hence to follow the same reduced form as for Einstein's equations in section(8.2.2). We present the quantity $Q \equiv -\mathcal{L}_{\vec{n}}\phi = \frac{\partial \phi}{\partial t}$ [cf. Eq.(3.89)] reducing the KG wave equation to first order in time and second order in space:

\begin{equation}

    \partial_{t}(\sqrt{\gamma}Q) - \partial_{i}(\beta^{i} \sqrt{\gamma}Q) + \partial_{i}(\alpha \sqrt{\gamma}\,\tensor{\gamma}{^i^j}\partial_{j}\phi) = \alpha \sqrt{\gamma}\frac{dV}{d|\phi|^{2}}\phi

\end{equation}

It's now time to enforce any assumed symmetries to get one step closer to solving for boson stars.

Although the boson star is found by a harmonic ansatz and is time-independent. We first choose to keep the full time-dependence. However we consider the spacetime to be \textit{sperichally symmetric.} The most generic metric in this case in terms of spherical coordinates is written as

\begin{equation}

    \boxed{ds^{2} = (-\alpha^2+a^{2}\beta^{2})dt^{2} + 2a^{2}\beta dt dr + a^{2} dr^{2} + r^{2}b^{2} d\Omega^{2}}

\end{equation}

where $d\Omega^{2}$ is the metric of a unit two-sphere, $r^{2} b^{2}d\Omega = r^{2} b^{2}d\theta^{2} + r^{2} b^{2} \sin{\theta}^{2}d\phi^{2}$.

\newline As encountered before $\alpha(t,r)$ is the lapse function, $\beta(t,r)$ is the radial component of the shift vector and $a(t,r),b(t,r)$ are components of the spatial metric above.

Following this metric, it results in the extrinsic curvature to only have two independent components $\tensor{K}{^i_j} = \textrm{diag}\big(\tensor{K}{^r_r},\tensor{K}{^\theta_\theta},\tensor{K}{^\theta_\theta} \big)$. 

\smallskip

\newline \textit{From now on, I am only going to present results derived from Mathematica and for the code used to generate these solutions please refer to the appendix.}

\smallskip

\newline By imposing the general metric given in Eq.(8.25), the constraint equations (8.15) and (8.14) become

\begin{align}

    &\frac{-2}{arb}\big\{\partial_{r}\Big[\frac{\parial_{r}(rb)}{a} \Big] + \frac{1}{rb} \Big[\partial_{r}\Big(\frac{rb}{a}\partial_{r}(rb)\Big)-a\Big]  \big\} +4\tensor{K}{^r_r}\tensor{K}{^\theta_\theta} + 2\tensor{K}{^\theta_\theta} = \notag \\

    &8 \pi G \Big[\frac{1}{a^2}\partial_{r}^{2}|\phi|^{2} + \frac{1}{\alpha^{2}}\big(\partial_{t}^{2}|\phi|^{2} + \beta^{2}\partial_{r}^{2}|\phi|^{2}-2\beta\partial_{t} \, 

    \partial_{r}|\phi|^{2} \big) +V \Big] \\

    \notag \\

    &2\big\{\partial_{r}\tensor{K}{^\theta_\theta} + \frac{1}{2}\Big( \frac{\partial_{r}(rb)}{rb} (\tensor{K}{^\theta_\theta}-\tensor{K}{^r_r}) \Big) \footnote{Note that there is a factor of $1/2$ difference in my solutions compared to the source, Dynamical boson stars by Steven L. Liebling} \big\} = \notag \\

    &4 \pi G \Big[\frac{1}{\alpha} \big(\partial_{t}\overline{\phi} \, \partial_{r}\phi + \partial_{t}\phi \, \partial_{r}\overline{\phi}\big) -\frac{2\beta}{\alpha}\big(\partial_{r}^{2}|\phi|^{2} \big) \Big]

\end{align}

respectively. For deriving Eq.(8.27) don't forget about the negative sign hiding in our $S_{i}$.

\newline Next the evolution equations defined in section (8.2.2) for the metric and extrinsic curvature components reduce to

\begin{align}

    &\textrm{the r-component of Eq.(8.12)}: \notag \\

    &\big(\partial_{t}-\mathcal{L}_{\beta} \big)\tensor{\gamma}{_i_j} = -2\alpha\tensor{K}{_i_j} \notag \\

    &\Rightarrow \frac{2}{a}(\partial_{t}a) -\frac{2}{a}\big(\beta \partial_{r}a + a\partial_{r}\beta \big) = -2\alpha \tensor{K}{_r^r} \notag \\

    &\boxed{\Rightarrow\partial_{t}a = \partial_{r}(a\beta) - \alpha a \tensor{K}{_r^r}} \\

    \notag \\

    &\textrm{the $\theta$-component of Eq.(8.12)}: \notag \\

    &\Rightarrow -\frac{2\beta}{r} - \frac{2\beta(\partial_{r}b)}{b} + \frac{2 \partial_{t}b}{b} = -2\alpha \tensor{K}{_\theta^\theta} \notag \\

    &\Rightarrow \frac{r\partial_{t}(b) - b\beta -r\partial_{r}(b)\beta}{rb} = -\alpha \tensor{K}{^\theta_\theta} \notag \\

    &\Rightarrow \partial_{t}(b) - \frac{\beta}{r}b - \partial_{r}(b)\beta = -\alpha b \tensor{K}{_\theta^\theta} \notag \\

    &\Rightarrow \boxed{\partial_{t}(b) = \frac{\beta}{r}\partial_{r}(rb) - \alpha b \tensor{K}{_\theta^\theta}}

\end{align}

To derive the above equations, we refer back to matrix (8.21) and Eq.(7.38).

\smallskip \newline Finally we move on to solving for the full projection of the Einstein equation onto $\Sigma_{t}$ with the coupled metric. As well as the above evolution equation, it will take on two components $r,\theta$. We shall start with the right-hand side of Eq.(8.11) for the $r$-component:

\begin{align}

    \big(\partial_{t}-\mathcal{L}_{\beta} \big)\tensor{K}{_i_j} &= \partial_{t}\tensor{K}{_i_j} - \big(\beta^{l}\partial_{l}\tensor{K}{_i_j} + \tensor{K}{_l_j}\partial_{i}\beta^{l}+\tensor{K}{_i_l}\partial_{j}\beta^{l} \big) \; \; \;(\mbox{choose \{i,j\} = r and sum over l = \{r,$\theta$\}}) \notag \\

    &= \partial_{t}\tensor{K}{_r_r}-\underbrace{\beta^{r}}_{\beta}\partial_{r}\tensor{K}{_r_r}-\tensor{K}{_r_r}\partial_{r}\beta^{r}-2\tensor{K}{_r_r}\partial_{r}\beta^{r} -\underbrace{\beta^{\theta}}_{0}\partial_{\theta}\tensor{K}{_r_r}-2\underbrace{\tensor{K}{_\theta_r}}_{0}\partial_{r}\beta^{\theta} \notag \\

    &= \partial_{t}\tensor{K}{_r_r}-\beta\partial_{r}\tensor{K}{_r_r}-2\tensor{K}{_r_r}\partial_{r}\beta \; \; \; \; \; (\mbox{contract with $\tensor{\gamma}{^r^r}$}) \notag \\

    &= \partial_{t}\tensor{K}{_r^r} - \beta \Big[\partial_{r}\big(\tensor{\gamma}{^r^r} \tensor{K}{_r_r} \big) - \big(\partial_{r}\tensor{\gamma}{^r^r}\big)\tensor{K}{_r_r} \Big] - 2\tensor{K}{_r^r}\partial_{r}\beta \notag \\ 

    &= \partial_{t}\tensor{K}{_r^r} - \beta\partial_{r}\tensor{K}{_r^r} - 2\beta\Big(\frac{\partial_{r}a}{a^{3}}\Big)\tensor{K}{_r_r} - 2\tensor{K}{_r^r}\partial_{r}\beta \; \footnote{I get additional terms compared to the source, Dynamical boson stars by Steven L. Liebling} \notag

\end{align}

Where in the second line we have used $\tensor{g}{^i^t}\cdot \alpha^{2} = \beta^{i} $[cf. matrix (8.23) - can check in Mathematica solutions for explicit matrix.] and the fact that $\vec{K}$ is diagonal. In the final equality we substituted $\tensor{\gamma}{^r^r}= \frac{1}{a^2}$ to derive $\partial_{r}\tensor{\gamma}{^r^r}$

\smallskip \newline \textbf{Note} the reason we contract all the solutions with $\tensor{\gamma}{^i^j}$ is for the purpose of being consistent with the source's solutions.

\smallskip

\newline 

In order to derive the left-hand side of Eq.(8.11) please refer to the appendix for my Mathematica codes. Hence putting everything together one gets:

\begin{align}

    &\partial_{t}\tensor{K}{_r^r} - \beta\partial_{r}\tensor{K}{_r^r} - 2\beta\Big(\frac{\partial_{r}a}{a^{3}}\Big)\tensor{K}{_r_r} - 2\tensor{K}{_r^r}\partial_{r}\beta = -\frac{1}{a}\partial_{r}\Big(\frac{\partial_{r}\alpha}{a}\Big) + \notag \\ &\alpha \, \Big\{ -\frac{2}{arb} \partial_{r}\big[\frac{\partial_{r}(rb)}{a}\big] + \mbox{trK}\tensor{K}{_r^r} -2\tensor{K}{_r^r}\tensor{K}{_r^r} \footnote{} -\frac{4 \pi G}{a^2}\big[2\partial_{r}^{2}|\phi|^{2} + a^{2}V(|\phi|^{2}) \big]\Big\}

\end{align}

For the $\theta$-component we simply follow the same steps as we did for the $r$-component and consequently one sees: 

\begin{align}

    &\partial_{t}\tensor{K}{_\theta^\theta} - \beta\partial_{r}\tensor{K}{_\theta^\theta} = \frac{\alpha}{(rb)^{2}} - \frac{1}{a(rb)^2}\partial_{r}\Big[\frac{\alpha rb}{a}\partial_{r}(rb) \Big] - 2\alpha\tensor{K}{_\theta^\theta}\tensor{K}{_\theta^\theta} \footnote{} + 

    &\alpha \, \Big[\mbox{trK}\tensor{K}{_\theta^\theta} - 4\pi G V(|\phi|^{2}) \Big]

\end{align}

Moreover, by setting all the $w$-equations below to zero and plugging in for $\Phi$ and $\Pi$ in our $y$-equations, we'll indeed see that they will equate to zero, i.e. the $y$- equations are derived from the wave equation.

\begin{align*}

    \begin{cases}

    w_{1} =\Phi - \partial_{r}\phi  \\ w_{2}=  \Pi - \frac{a}{\alpha}(\partial_{t}\phi-\beta\partial_{r}\phi) \\ w_{3}=\square\phi   

    \end{cases}

    \begin{cases}

    y_{1} = \partial_{t}\phi - (\beta\Phi+\frac{\alpha}{a}\Pi) \\ y_{2} = \partial_{t}\Phi - \partial_{r}\Big(\beta\Phi+\frac{\alpha}{a}\Pi \Big) \\ y_{3} = \partial_{t}\Pi - \frac{1}{(rb)^{2}} \partial_{r}\Big[(rb)^{2} \big(\beta\Pi + \frac{\alpha}{a}\Phi \big) \Big] - 2 \Big[\alpha\tensor{K}{_\theta^\theta} - \beta\frac{\partial_{r}(rb)}{rb} \Big]\Pi + \alpha a \frac{dV}{d|\phi|^{2}}\phi

    \end{cases}

\end{align*}

 Hence after reducing the Klein-Gordon equation to first order in time and space, we get the following set of evolution equations

\begin{align}

    &\partial_{t}\phi = \beta\Phi+\frac{\alpha}{a}\Pi \\

    &\partial_{t}\Phi = \partial_{r}\Big(\beta\Phi+\frac{\alpha}{a}\Pi \Big) \\

    &\partial_{t}\Pi = \frac{1}{(rb)^{2}} \partial_{r}\Big[(rb)^{2} \big(\beta\Pi + \frac{\alpha}{a}\Phi \big) \Big] + 2 \Big[\alpha\tensor{K}{_\theta^\theta} - \beta\frac{\partial_{r}(rb)}{rb} \Big]\Pi - \alpha a \frac{dV}{d|\phi|^{2}}\phi

\end{align}

The set of equations, Eqs.(8.26)-(8.34), describe general time-dependant, spherically symmetric solutions in a metric-coupled complex scalar field. 



\section{Mini Boson stars}

In this chapter we aim to solve for a particular case of a boson star. 

\newline The concept behind a star is dependent on the configuration of matter which stays localised. Therefore one looks for a localised and time-independent arrangement of matter which is both stationary and regular everywhere. Such configuration does not exist in real scalar fields. However we can make use of the fact that the stress-energy tensor [cf. Eq.(8.3)] solely depends on the modulus of the scalar field and its gradients hence one can make the assumption of the scalar field being time-independent whilst still maintaining the time-independence of the gravitational field. 



\section{Conclusion}




\section{References and Bibliography}

\section{Appendix}


